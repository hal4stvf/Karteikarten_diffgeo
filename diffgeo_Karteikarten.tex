%
%		* ----------------------------------------------------------------------------
%		* "THE BEER-WARE LICENSE" (Revision 42/023):
%		* Ronny Bergmann <mail@rbergmann.info> wrote this file. As long as you retain
%		* this notice you can do whatever you want with this stuff. If we meet some day,
%		* and you think  this stuff is worth it, you can buy me a beer or a coffee in return. 
%		* ----------------------------------------------------------------------------
%
%
% Beispiel zur Dokumentvorlage für Din A6 Karteikarten 
% -- Version 1.8b --
%
%\documentclass[a7paper,8pt,grid=front,print]{kartei}
\documentclass[a6paper,11pt,grid=front]{kartei}
\usepackage[utf8]{inputenc}

\usepackage{amsfonts}
\usepackage{amsmath}
\usepackage{paralist}
\usepackage[ngerman]{babel}

\setlength{\parindent}{0pt}

\usepackage{tikz}
\usetikzlibrary{backgrounds}

\newcommand{\mtwos}[1]{\begin{displaymath}
 #1 \end{displaymath}}
\newcommand{\fl}[1]{\begin{flushleft}
 #1 \end{flushleft}}
\newcommand{\centr}[0]{\vspace*{\stretch{1}}}

\sloppy %für die Silbentrennung -> Führt zu einem "erzwungenen" Blocksatz
\newcommand{\R}{\mathbb{R}}
\newcommand{\N}{\mathbb{N}}
\newcommand{\Z}{\mathbb{Z}}
\newcommand{\C}{\mathbb{C}}
\newcommand{\K}{\mathbb{K}}
\newcommand{\T}{\mathbb{T}}

\newcommand{\Rp}{\mathbb{R}_{\geq 0}}
\newcommand{\hA}{\mathfrak{A}}
\newcommand{\gX}{{X}}
\newcommand{\gL}{\mathcal{L}}
\newcommand{\Oh}{\mathcal{O}}

\newcommand{\iv}[1]{#1^{-1}} 
\newcommand{\CA}{C^0(D,X)}
\newcommand{\ov}[1]{\overline{#1}}
\newcommand{\eps}{\varepsilon}
\newcommand{\norm}[1]{\left \|#1\right\| }
\renewcommand{\epsilon}{\varepsilon}

\newcommand{\pp}[1][i]{\frac{\partial}{\partial x_{#1}}\bigg|_p}
\newcommand{\pd}[1][x_i]{\frac{\partial}{\partial {#1}}}
\newcommand{\blf}[1]{\langle #1 \rangle}
\newcommand{\nabladt}[1][t]{\frac{\nabla}{d#1}}
\newcommand{\nablapt}[1][t]{\frac{\nabla}{\partial#1}}

\newcounter{def}
\setcounter{def}{1}
\newcounter{satz}
\setcounter{satz}{1}

\newcommand{\defreset}{\setcounter{def}{1}}
\newcommand{\satzreset}{\setcounter{satz}{1}}
\newcommand{\thisdef}{\thedef\ \stepcounter{def}}
\newcommand{\thissatz}{\thesatz\ \stepcounter{satz}}
\begin{document}

%%Neudefinition der Karten
\antwort{} %führt zu (k)einem Text auf der Rückseite
\renewcommand{\theCardID}{}
\setheadsepline{0pt}[\color{black}] 
\renewcommand{\fachstil}{\textsc}
\renewcommand{\kommentarstil}{\textsc}
%%% Hauptbefehl um eventuell später auf ein anderes
%% Format um zuschalten.
%																				Typ Name  Vorderseite
\newcommand{\nonameyet}[4]{\begin{karte}[{#1} ({#2})]{{#3}} #4\end{karte}}
%%%%%%%%%%%%%%%%%%%%%%%%%%%%%%%%%%%%%%%%%%%%%%%%%%%%%%%%%%%%%%%%%%%%%%%%%%%%%%%%
%%%%%%%%%%%%%%%%%%%%%%%%%%%%%%%%%%%%%%%%%%%%%%%%%%%%%%%%%%%%%%%%%%%%%%%%%%%%%%%%

\kommentar{1.2 Differenzierbare Mf}
\nonameyet
{Definition} {topologische Mannigfaltikeit}
{Sei $M$ topologische Raum mit 
\\ $M \not = \emptyset$,
$M$ erfüllt zweites Abzählbarkeitsaxiom und ist Hausdorffsch.
\\
~\\
Dann heißt $M$ topologische Mannigfaltikeit der Dimension $n$, falls es zu
jedem Punkt $p \in M$ eine in $M$ offene Umgebung $U$ 
und eine offene Teilmenge $V\subset \R^n$ gibt, 
so dass ein Homöomorphismus $\varphi : U \to V$ existiert.
\\
$(U,\varphi)$ nennt man eine Karte von $M$.
}
{}

\nonameyet
{Defintion} {$C^\infty$-Atlas $\mathcal{A}$}
{Sei $M$ top. Mf. und $\mathcal{A} 
= \{(U_\alpha,\varphi_\alpha) | \alpha \in A \}$
eine Familie von Karten von M. 
\\
Dann heißt $\mathcal{A}$ ein  \texttt{$(C^\infty)$-Atlas}, falls folgendes gilt:
\begin{enumerate}[1.]
\item $\bigcup_{\alpha \in A} U_\alpha = M$
\item $\varphi_\beta \circ \varphi_\alpha^ {-1} 
: \varphi_\alpha(U_\alpha \cap U_\beta) \to \varphi_\beta(U_\alpha \cap U_\beta$
ist $C^\infty$ für alle $\alpha, \beta \in A$. 
\footnotesize
(Die Abbildungen $\varphi_\beta \circ \varphi_\alpha^{-1}$ heißen 
\texttt{Karten- oder Koordinatenwechsel}.)
\end{enumerate}
}
{}

\nonameyet
{Definiton} {$C^\infty$-differenzierbare Struktur}
{Sei $\mathcal{A}$ ein $C^\infty$-Atlas. 
%
\\ 
\fl{
Dann heißt $\mathcal{A}$ eine \texttt{$(C^\infty)$-differenzierbare Struktur},
falls folgendes erfüllt ist:
}
%
\begin{enumerate}[1.]
\setcounter{enumi}{2}
%
\item $\mathcal{A}$ ist maximal in dem Sinne, dass eine Karte $(U,\varphi)$ 
bereits zu $\mathcal{A}$ gehört, falls 
%
\vspace{-0.5em}
\[
\begin{aligned}
%%
\varphi \circ \varphi_\alpha^{-1}: \varphi_\alpha(U \cap U_\alpha) \to 
\varphi (U \cap U_\alpha)
\\ \text{ und }
\varphi_\alpha \circ \varphi^{-1}: \varphi(U \cap U_\alpha) \to
\varphi_\alpha (U \cap U_\alpha)
%%
\end{aligned}
\]
%
für alle $\alpha \in A$  $C^\infty$ ist.
\end{enumerate}
}
{}

\nonameyet
{Defintion} {$n$-dim differenzierbare Mannigfaltigkeit}
{Sei $M$ eine $n$-dim. topologische Mf und $\mathcal{A}$ eine differenzierbare Struktur.
\\
~\\
Eine \texttt{$n$-dimensionale Mannigfaltigkeit} ist ein Paar $(M,\mathcal{A})$.
}
{}
\nonameyet
{Definition} {Untermannigfaltigkeit}
{
Sei $M \subset \R^k$ eine nichtleere Teilmenge.
\\
\fl{$M$ heißt \texttt{$n$-dim. Untermannigfaltigkeit von $\R^k$}, wenn es
}
\begin{enumerate}[1.]
\item zu jedem Punkt $p\in M$ eine offene Umgebung $U \subset \R^k$  und
\item einen Diffeomorphismus $\varphi : U \to V \subset \R^k$ gibt, so dass
\[
\varphi(M \cap U) = \{ x = (x_i) \in \R^k | x_{n+1} = \dots = x_k = 0
\} \cap V.
\]
\end{enumerate}
%
\tiny
Man kann eine Untermannigfaltigkeit des $\R^k$ auch als differenzierbare Mf
auffassen. 
% TODO
}
{}

%%%%%%%%%%%%%%%%%%%%%%%%%%%%%%%%%%%%%%%%%%%%%%%%%%%%%%%%%%%%%%%%%%%%%%%%%%%%%%%%
% 1.3
%%%%%%%%%%%%%%%%%%%%%%%%%%%%%%%%%%%%%%%%%%%%%%%%%%%%%%%%%%%%%%%%%%%%%%%%%%%%%%%%
\kommentar{1.3 Differenzierbare Abbildungen}
\nonameyet
{Definition} {differenzierbare Abbildung zw Mf}
{
Seien $(M,\{(U_\alpha, \varphi_\alpha)| \alpha \in A\})$ 
und $(N,\{(V_\beta, \psi_\beta)| \beta \in B\})$ 
differenzierbare Mf und $f: M\to N$ stetig.
\\
\fl{ Dann heißt $f$ \texttt{differenzierbar} oder auch \texttt{glatt}, wenn}
\[
\psi_\beta \circ f \circ \varphi_\alpha^{-1}:
\varphi_\alpha(U_\alpha \cap f^ {-1}(V_\beta)) \to 
\psi_\beta(V_\beta)
\]
für alle $\alpha\in A,\beta \in B$ $C^\infty$ ist.
\\
\fl{Wir setzen:}
$C^\infty := \{ f: M\to N| f \text{ ist } C^\infty\}$
}
{}

\nonameyet
{Definition} {Diffeomorphismus}
{
Sei $f: M \to N$ eine bijektive Abbildung.
\\
~\\
$f$ heißt ein \texttt{Diffeomorphismus}, falls $f$ und $f^{-1}$ differenziert
sind.
}
{}

\nonameyet
{Definition} {diffeomorph}
{
Existiert ein Diffeomorphismus zwischen zwei differenzierbaren Mf $M$ und $N$,
so heißen $M$ und $N$ diffeomorph.
}
{}


%%%%%%%%%%%%%%%%%%%%%%%%%%%%%%%%%%%%%%%%%%%%%%%%%%%%%%%%%%%%%%%%%%%%%%%%%%%%%%%%
% 1.4 Der Tangentialraum
%%%%%%%%%%%%%%%%%%%%%%%%%%%%%%%%%%%%%%%%%%%%%%%%%%%%%%%%%%%%%%%%%%%%%%%%%%%%%%%%
\kommentar{Der Tangentialraum}


\nonameyet
{Definition} {Funktionenkeime}
{
\footnotesize
Es seien $U_i\subset M$, $p \in U_i$, $i = 1,2$, $f_i \in C^\infty(U_i,\R)$
beliebig.
%
\fl{Wir definieren für beliebig $i,j = 1,2$ und $f{_i,_j}$ wie oben eine
Äquivalenzrelation }
\[
f_i \sim f_j: \Leftrightarrow \exists V \subset M, p\in V: f_i|_V = f_j|_V
\]
\fl{Nun definieren wir folgendene Menge}
\[
\mathcal{F}_p := \{f: U \to \R | U \subset M \text{offen}, p \in U,
f \text{differenzierbar}\}/ \sim
\]
\fl{und bezeichnen ihre Elemente als \texttt{Funktionskeime} und schreiben
für $f\in C^\infty(U,\R)$, $p\in U$ für den Funktionskeim $[f]$.}
%
\scriptsize
\fl{\texttt{Bemerkung:}  $\mathcal{F}_p$ ist eine $\R$-Algebra mit }
\[
[f] + [g] := [f + g], [f]\cdot[g] := [fg]
\]
\fl{Zudem ist }
\[
v: \mathcal{F}_p \to \R, [f] \mapsto v([f]) := f(p)
\]
\fl{ist wohldefiniert. Man kann ein Funktionskeim aber in keinen anderen Punkt 
außer $p$ auswerten.}
}
{}


\nonameyet
{Definition} {Tangentialvektor}
{
Es sei $M$ eine diffb. Mf und sei
\[
v : \mathcal{F}_p \to \R,
\]
\fl{eine lineare Abbildung, die die sogenannte Leibnis-Regel erfüllt, d.h.}
\[
v([f]\cdot[g]) = v([f])\cdot g(p) + f(p)\cdot v([g]).
\]
\fl{Dann nennen wir $v$ einen \texttt{Tangentialvektor an $M$ von $p$}.}
}
{}


\nonameyet
{Definition} {Tangentialraum}
{
Die Menge
\[
T_pM := \{ v \text { ist Tangentialvektor von $M$ in $p$}\}
\]
versehen mit der Vektorraumstuktur
\[
(v+w)[f] := v([f]) + w([f]), \quad (\alpha v)[f] := \alpha \cdot v([f])
\]
\fl{heißt \texttt{Tangentialraum von $M$ in $p$}.}
%\\
\fl{Ist $f\in C^\infty(U,\R)$, so schreiben wir $v(f) := v([f])$}.
}
{}


\nonameyet
{Bemerkung 1.4.2} {}
{
Per Hand
}
{}

\nonameyet
{Definition} {lokale und Standardkoordinaten}
{
Sei $M$ differenzierbar Mf, $(U,\varphi)$ eine Karte um $p \in U$:
$\varphi: U \to \R^n$. 
\fl{Ist $u_1,\dots, u_n$, so, dass}
\[
u_i : \R^n \to \R, \quad u_i(v_1,\dots,v_n) = v_i
\]
\fl{ist, so heißen die $u_1,\dots,u_n$ \texttt{Standardkoordinaten von 
$\R^n$}.}
%
\fl{Nun definieren wir durch}
\[
x_i := u_i \circ \varphi, \quad \text{also} \quad 
\varphi = (x_1,\dots,x_n)
\]
\fl{die \texttt{lokalen Koordinaten} $x_1,\dots,x_n$.}
}
{}

\nonameyet
{Lemma} {\tiny Partielle Ableitung sind Tangentialvektoren}
{
\footnotesize
Es $M$ diffb. Mf, $(U,\varphi)$ eine Karte um p mit lokalen Koordinaten 
$\varphi= (x_1,\dots,x_n)$.
\fl{Wir definieren}
\[
\frac{\partial}{\partial x_i}\bigg|_p : \mathcal{F} \to \R,
\quad
[f] \mapsto \frac{\partial}{\partial x_i}\bigg|_p [f] 
:= \frac{\partial f }{\partial x_i}\bigg|_p
:= \frac{\partial f }{\partial x_i}(p)
:= \frac{\partial (f\circ \varphi^{-1})}{\partial u_i} (\varphi(p)).
\]
\fl{Nun gilt: $\frac{\partial}{\partial x_i}\bigg|_p$ ist ein Tangentialvektor an $p$.}	
TODO: hier auch Bsp 1.4.5
}
{}
\nonameyet
{Satz} {}
{
Sei $M$ eine $n$-dim diffb. Mf, $p\in M$.
\fl{Dann ist }
\[
\{\pp[1],\dots,\pp[n]\}
\]
\fl{eine Basis von $T_pM$ und für $v \in T_p M$ gilt:}
\[
v = \sum_{i=1}^n v(x_i) \cdot \pp.
\]
\fl{Es folgt daraus $\dim(M) = \dim(T_pM)$.}
}
{}

%%%%%%%%%%%%%%%%%%%%%%%%%%%%%%%%%%%%%%%%%%%%%%%%%%%%%%%%%%%%%%%%%%%%%%%%%%%%%%%
% 1.5 Das Differential einer Abbildung
%%%%%%%%%%%%%%%%%%%%%%%%%%%%%%%%%%%%%%%%%%%%%%%%%%%%%%%%%%%%%%%%%%%%%%%%%%%%%%%
\kommentar{Das Differential einer Abbildung}

\nonameyet
{Definition} {Differential}
{
Sei $M$ diffb. Mf, $p\in M$, $F : M \to N$ differenzierbar.
\fl{Die lineare Abbildung}
\[
dF_p : T_pM \to T_{F(p)}N 
\] 
\fl{gegeben durch}
\[
dF_p(v)(f) := v(f\circ F)
\]
\fl{nennen wir das \texttt{Differential von $F$ in $p$}}.
}
{}

\nonameyet
{Satz} {\tiny Jacobimatrix und Differential}
{
\small
Sei $M$,$N$ $n$-dim bzw. $m$-dim diffb. Mf, $p\in M$, $F \in C^\infty(M,N)$. 
Weiter sei \begin{itemize}[-]
\item $(U,\varphi)$ Karte um $p$ mit lokalen Koordinanten 
$\varphi = (x_1,\dots,x_n)$ und  
\item $(V,\psi)$ Karte um $F(p)$ mit 
lokalen Koordinaten $\psi = (y_1,\dots,y_m)$.
\end{itemize}
\fl{Dann gilt:}
Die Matrix von $dF_p$ bzgl der Basen 
$(\frac{\partial}{\partial x_i}\big|_{p})$ und 
$(\frac{\partial}{\partial y_i}\big|_{F(p)})$ ist gleich der Jacobimatrix von
$\psi \circ F \circ \varphi^ {-1}$ in $\varphi(p)$.
}
{}

\nonameyet
{Satz} {Kettenregel}
{
Sei $M$,$N$,$L$ differenzierbare Mf, $p\in M$ und $F\in C^\infty(M,N)$ und $G\in C^\infty(N,L)$.
\fl{Dann gilt:}
\[
d(G\circ F)_p = dG_{F(p)}\circ dF_p 
\]
}
{}


\nonameyet
{Definition und Bsp} {Kurven}
{
Sei $M$ differenzierbare Mf, $a,b \in \R$, $I = (a,b)$.
\\
Zu $c\in C^\infty(I, M)$ sagen wir auch \texttt{glatte Kurve in M}.
\fl{Weiter definiere wir für $t\in (a,b)$:}
\[
c'(t) = \dot c(t) := dc_t(\frac{\partial}{\partial x}\bigg|_t)
\in T_{c(t)} M.
\]
\fl{Mit $c: [a,b] \to M$ meinen wir :}
\[
\exists \eps > 0,\; \overline c :(a - \eps, b + \eps) \to M
\quad : \quad  \overline c |_{(a,b)} = c
\]
}
{}

\nonameyet
{Satz} {Kettenregel und Kurven}
{
Sei $M$ differenzierbare Mf, $F\in C^\infty(M,N)$, $p\in M$ und $v\in T_pM$.
Weiter sei $I = (a,b)$, $0\in I$ und $c \in C^\infty(I,M)$ mit $c(0) = p$ 
und $c'(0) = v$.
\fl{Dann gilt:}
\[
dF_p(v) = (F \circ c)' (0).
\]
}
{}

%%%%%%%%%%%%%%%%%%%%%%%%%%%%%%%%%%%%%%%%%%%%%%%%%%%%%%%%%%%%%%%%%%%%%%%%%%%%%%%%
% 1.6 Resultate aus der Analysis
%%%%%%%%%%%%%%%%%%%%%%%%%%%%%%%%%%%%%%%%%%%%%%%%%%%%%%%%%%%%%%%%%%%%%%%%%%%%%%%%
\nonameyet
{Definition 1.6.1} {\small (differenzierbare) Zerlegung der Eins}
{
	\scriptsize
Sei $M$ differenzierbare Mannigfaltigkeit. $I,A$ beliebige Indexmengen,
$\varphi_i \in C^\infty(M)$ für alle $i\in I$ und $(\varphi_i)_{i\in I}$ 
eine Familie. $\mathcal{U} = \{U_\alpha | \alpha \in A\}$ sei eine offene 
Überdeckung von $M$.
\fl{Gilt: }
\begin{enumerate}[1.]
\item die Träger der $\varphi_i$ sind für alle $i\in I$ lokal endlich, d.h.
\[
\forall p\in M \exists U\subset M, p\in U: U \cap \text{supp}\:\varphi_i \neq \emptyset
\quad \text{für höchstens endliche viele $i \in I$}
\]
%%
\item Summe der Funktionenswerte ist $1$ in jedem Punkt, genauer:
%\vspace{-0.8em}
\[
\sum_{i\in I} \varphi_i(p) = 1 \forall p\in M 
\quad \text{ und } \quad
\varphi_i(p) \geq 0 \forall p\in M,\forall i \in I,
\]
\end{enumerate}
\fl{so heißt die Familie $(\varphi_i)_{i\in I}$ 
\texttt{eine Zerlegung des Eines von $M$}.}
%
\fl{Gilt:}
\[
\forall i \in I \; \exists \alpha \in A: \text{supp }\varphi_i \subset U_\alpha,
\]
\fl{so heißt die Familie $(\varphi_i)_{i\in I}$ der \texttt{Überdeckung 
$\mathcal{U}$ untergeordnet}.}
}
{}

\nonameyet
{Satz 1.6.2} {Existenz der Zerlegung der Eins}
{
	\small
Sei $M$ diffb Mf, $\mathcal{U}$ eine offene Überdeckung von $M$.
\fl{Dann existiert eine abzählbare differenzierbare Zerlegung der Eins,
der $\mathcal{U}$ untergeordnet ist.}
\fl{\textbf{Korollar:}} 
Sei $U \subset M$ offen, $A \subset U$ abgeschlossen in $M$, $f\in C^\infty(U)$
\fl{Dann gibt $g\in C^\infty(M)$ mit $g|_A = f|_A$ und $g|_{M\setminus U} = 0$.}
}
{}

\nonameyet
{Satz} {lokale Umkehrsatz}
{
Sei $U\subset \R^n$ offen, $f\in C^1(U,\R^n)$
\fl{Ist $df_p : \R^n \to \R^n$ für ein $p \in U$ invertierbar, dann gilt}
\begin{enumerate}[1.]
\item es gibt eine Umgebung $V$ von $p$ und $W$ von $f(p)$, 
so dass $f|_V : V \to W$ in Diffeomorphismus ist. 
\item Das Differential von $f^ {-1}$ in $q \in W$ ist gegeben durch
\[
(df^{-1})_q = (df_{f^{-1}(q)})^{-1}
\]
\end{enumerate}
%\small
Dieser Satz gilt auch für $f\in C^\infty(U,\R^n)$.
}
{}

\nonameyet
{Satz 1.6.5} {}
{
Seien $M,N$ differenzierbare Mf gleicher Dimension. $U \subset M$ offen, 
$f\in C^\infty(U,N)$.
\fl{Existiert ein $p\in U$, so dass $df_p : T_pM \to T_{f(p)}N$ invertierbar
ist,
so existiert eine Umgebung $V$ um $p$ und $W\subset N$ um $f(p)$, so dass }
\[
f|_V : V \to W
\]
ein Diffeomorphismus ist.
}
{}


\nonameyet
{Satz 1.6.6} {}
{
Sei $U\subset \R^n$ eine Umgebung von $0\in \R^n$. $f\in C^\infty(U,\R^k)$ mit
$f(0) = 0 $. Sei $df_0$ 
\begin{enumerate}[1.]
\item Ist $n \leq k$ und das Differential $df_0$ injektiv, so gibt es ein
Diffeomorphismus $\psi : $
\end{enumerate}
TODO
}
{}

%%%%%%%%%%%%%%%%%%%%%%%%%%%%%%%%%%%%%%%%%%%%%%%%%%%%%%%%%%%%%%%%%%%%%%%%%%%%%%%
% 1.7 Untermannigfaltigkeiten
%%%%%%%%%%%%%%%%%%%%%%%%%%%%%%%%%%%%%%%%%%%%%%%%%%%%%%%%%%%%%%%%%%%%%%%%%%%%%%%
\kommentar{\small Unter-Mf}

\nonameyet
{Definition 1} {Immersion}
{
$M,N$ seien differenzierbare Mf. $F \in C^\infty(M,N)$. 
\fl{Wenn $df_p: T_pM \to T_{f(p)}N$ für alle $p\in M$ injektiv ist, so 
heißt $F$ eine \texttt{Immersion}.}
\fl{Weiter nennen wir $F(M)$ eine \texttt{immersierte Untermannigfaltigkeit
von $N$}.}
}
{}
\newcommand{\point}{\mathbf{\cdot}}
\nonameyet
{Definition 2} {Einbettung}
{
 $M,N$ seien differenzierbare MF. 
\\
$F \in C^\infty(M,N)$ sei eine Immersion und 
injektiv. 
\\
$F(M)$ sei mit der Teilraumtopologie ausgestattet. 

\fl{Falls $F : M \to F(M) \subset N$ ein Homöomorphismus ist, so heißt 
$F$ eine \texttt{Einbettung}.

\fl{$F(M)$ heißt dann \texttt{eingebettete Untermannigfaltigkeit von $N$}.}
} 
}
{}

\nonameyet
{Satz 1 und Definition 3} {}
{
$M$,$N$ seien $n$- bzw. $k$-dim. diffb. Mf,
\\
$F : M \to N$ eine Immersion und $p \in M$.
\\

\fl{Dann gibt es eine Umgebung $U$ von $p$ in M und eine Karte $(V,\psi)$
von $N$ um $F(p)$, wobei $\psi = (y_1,\dots,y_k)$, so dass

\begin{enumerate}[1.]
\item $y_{n+1}(q) = \dots = y_{k}(q) = 0$ für alle $q \in V \cap F(U)$ und 
\item $F|_U$ ist eine Einbettung.
\end{enumerate}

\fl{Die Karte $(V,\psi)$ heißt \texttt{Untermannigfaltigkeitskarte}.}
}
}
{}

\nonameyet
{Bemerkung 1 und 2,Def 3} {Unter-Mf, Einschränkung}
{
\small
	Seien $M,N,P$ diffb. Mf.
\begin{enumerate}[1.]
\item Satz 1 liefert als Spezialfall: Für $F : M \to \R^k$ ist $F(M)$ 
gerade eine Untermannigfaltigkeit von $\R^k$ im Sinne der früheren Definition.
(Kapitel 1.2, Begriff der differenzierbaren Mf) 

\item Einschränkung des Abbildungsraum: Für $F\in C^\infty(M,N)$, $i : P \to M$
eine Einbettung. 
\par
Dann heißt $F\circ i \in C^\infty(P,N)$ die \texttt{Einschränkung von $F$
auf $P$}.
\par Man schreibt auch $F|_P : P \to N$. Dabei ist wegen $i(P) \subset M$, dies
als Einschränkung des Abbildungsraum zu interpretieren.
\end{enumerate}
}
{}

\nonameyet
{Satz 2} {Einschränkung des Zielraums}
{
$M,N,P$ seien diffb Mf, $F\in C^\infty(M,N)$, $i : P \to N$ eine Einbettung.
\\
Es sei weiter: $F(M) \subset i(P)$ 
und $G : M \to P$ durch $F(p) = i(G(p))$ definiert
$^{\text{(wohldefiniert, da $i$ injektiv)}}$.
\fl{Dann gilt:}
\begin{enumerate}[1.]
\item Falls $i$ eine Einbettung ist, so ist $G$ stetig. 
\item Falls $G$ stetig ist, so ist $G$ glatt.
\end{enumerate}
}
{}

\nonameyet
{Definition 4} {regulärer Punkt, kritischer Punkt}
{
$M,N$ differenzierbare Mf, $F\in C^\infty(M,N)$.

\fl{Ist für $p\in M$ das Differential $df_p : T_pM \to T_{F(p)}N$ surjektiv,
dann heißt $p$ \texttt{regulärer Punkt}.\par
Andernfalls \texttt{kritischer Punkt}.}
\fl{
Sind für $q\in N$ alle Punkte $p \in F^ {-1}(q)$ regulär, so heißt $q$ 
\texttt{regulärer Wert}.	
\par
Andernfalls \texttt{kritischer Wert}.}
}
{}

\nonameyet
{Satz 3} {}
{
\small
Es seien $M$, $N$ $n$- bzw. $k$-dim. Mf. $F\in C^\infty(M,N)$, $q\in F(M)$ ein
regulärer Wert. Es sei $F^ {-1}(q)\subset M$ versehen mit Teilraumtopologie.
\fl{Dann gilt:}
\begin{enumerate}[1.]
\item $F^ {-1}(q)$ ist eine $n-k$-dim. topologische Mf. 
\item Es existiert eine eindeutige differenzierbare Struktur auf $F^ {-1}(q)$, 
so dass $i : F^ {-1}(q) \to M$ ein Einbettung ist und damit insbesondere:
\par $i(F^ {-1}(q))$ ist eine eingebettete Unter-Mf von $M$.
\end{enumerate}
}
{}

\nonameyet
{Satz 4} {}
{
$N$ seien diffb. Mf. Es sei $M \subset N$ versehen mit der Teilraumtopologie
eine topologische Mf.
\fl{Dann gilt:}
\fl{Trägt $M$ eine differenzierbare Sturktur bezüglich derer $i: M \to N$ eine 
		Einbettung ist, so ist diese differenzierbare Struktur eindeutig.}
}
{}

\nonameyet
{Bsp und Bem} {}
{
2 Bsp zum Satz vom regulären Wert.
\fl{und folgende Bem:}
TODO, Auch falls $M \subset N$ gilt , ist $T_pM$ nicht auf natürlicheweise
ein Unterraum von $T_pN$. 
Betrachte: $di_p(T_pM) \subset N$. 
\\
TODO wie identifiziert man $T_pM$ und $di_p(T_pM)$.?
}
{}

%%%%%%%%%%%%%%%%%%%%%%%%%%%%%%%%%%%%%%%%%%%%%%%%%%%%%%%%%%%%%%%%%%%%%%%%%%%%%%%
% 1.8 Der Tangentialbündel 
%%%%%%%%%%%%%%%%%%%%%%%%%%%%%%%%%%%%%%%%%%%%%%%%%%%%%%%%%%%%%%%%%%%%%%%%%%%%%%%
\kommentar{1.8 Tangentialbündel}

\nonameyet
{Defintion 1} {$(C^\infty)$-Vektorbündel}
{
\scriptsize
Sei $M$, $E$ diffb Mf. 
Für eine $U \subset M$ sei
$pr_1 : U \times \R^k \to U, (u,x_1,\dots,x_k) \mapsto u$.
\\
Es sei $\pi\in C^\infty(E,M)$ und surjektiv.
\fl{Falls für alle $p\in M$ gilt:}
\begin{enumerate}[1.]
\item $E_p := \pi^ {-1}(p)$ ist ein $k$-dimensionaler Vektorraum.  
\item Es existiert eine Umgebung $U\subset M$ von $p$ und ein 
Diffeomorphismus: 
\[
\varphi: \pi^ {-1}(U) \to U \times \R^k,
\]
so dass gilt:
\[
\pi = pr_1 \circ \varphi
\text{ und }
\varphi|_{E_q}: E_q \to \{ q\} \times \R^k \overset{\sim}{=} \R^k 
\text{ ist linear.}
\]
\end{enumerate}

\fl{Dann heißt das Paar $(E,\pi)$ ein \texttt{$(C^\infty)$-Vektorbündel vom 
Rang $k$ über $M$}.}

\tiny
\begin{itemize}[-]
\item Das Urbild eines(!) Punktes heißt auch Faser, also: 
$E_p := \pi^ {-1}(p)$ heißt \texttt{Faser von $p$}. 
\item $E$ heißt \texttt{Totalraum.}
\item $M$ heißt \texttt{Basis des Vektorbündels $E$}
\item $\varphi$ heißt \texttt{lokale Trivialisierung}.
\end{itemize}
}
{}
\nonameyet
{Bem 1.8.2} {}
{
TODO Bem. 1.8.2
}
{}

\nonameyet
{Definition 1 und Notiz} {Schnitt}
{
	\small
Sei $M$,$E$ diffb Mf. $(E,\pi)$ ein $C^\infty$-Vektorbündel. 
Es sei $s \in C^\infty(M,E)$.

\fl{Gilt:}
\[
\pi \circ s = id_M, \text{ also }
\pi(s(p)) = p \quad \forall p \in M,
\]

\fl{so heißt $s$ ein \texttt{Schnitt von $E$}.}
\fl{Die Menge aller Schnitte von $E$ wird mit $\Gamma(E)$ bezeichnet.}
\fl{Ist $s$ nur auf einer offenen Teilmenge definiert, so spricht man von
einem \texttt{lokalen Schnitt von $E$}.}

\fl{Notiz: $\Gamma(E)$ ist ein $\R$-Vektorraum. Zudem ist für $s \in \Gamma(E)$,
$f\in C^\infty(E)$, $(fs)(p) := f(p)\cdot s(p)$ auch $fs \in \Gamma(E)$. Wobei 
diese skalare Multiplikation in $E_p$ zuverstehen ist. 
\par
Algebraisch ist $\Gamma(E)$ ein Modul über dem Ring $C^\infty(M)$.
}
}
{}

\nonameyet
{Satz und Definition} {Tangentialbündel.}
{
TODO
}
{}

%%%%%%%%%%%%%%%%%%%%%%%%%%%%%%%%%%%%%%%%%%%%%%%%%%%%%%%%%%%%%%%%%%%%%%%%%%%%%%%%
% 1.9 Vektorfelder
%%%%%%%%%%%%%%%%%%%%%%%%%%%%%%%%%%%%%%%%%%%%%%%%%%%%%%%%%%%%%%%%%%%%%%%%%%%%%%%%
\kommentar{1.9 Vektorfelder}

\nonameyet
{Definition 1} {Vektorfelder}
{
Elemente in $\Gamma(TM)$ heißen \texttt{(differenzierbare) Vektorfelder auf $M$.}
\fl{Für $U \subset M$ offen und $X\in \Gamma(TU)$ spricht man von 
\texttt{lokalen Vektorfeldern auf $M$}.}
\fl{Sei $X\in \Gamma(TM)$, $p\in M$, $f\in C^\infty(M)$, so definieren wir }
\[
X(f) : M \to \R, \quad p \mapsto X_p(f).
\]
\fl{$X(f)$ nennt man \texttt{Richtungsableitung} ???}
\fl{TODO Bsp $(\frac{\partial}{\partial x_i})$} für eine Vektorfeld.
}
{}

\nonameyet
{Satz 1.9.4/5} {}
{
TODO
\\
\fl{$\Gamma(TU)$ ist ein freier Modul über $C^\infty(U)$ mit Basis
$\{\frac{\partial}{\partial x_1},\dots,\frac{\partial}{\partial x_n}\}$}
\fl{$\Gamma(TM)$ ist ein Modul über $C^\infty(M)$}
\fl{$X\in \Gamma(TM)$ sind Derivationen auf $C^\infty(M)$.}
\[
X(fg) = X(f)g + fX(g).
\]
}
{}

\nonameyet
{Definition 1.9.7} {Lieklammer, Liealgebra}
{
Sei $V$ ein $K$-VR und $[\cdot,\cdot]: V\times V \to V$ eine Abbildung,
für die gilt:
\begin{enumerate}[1.]
\item bilinear
\item antisymmetrisch $([v,w] = - [w,v])$
\item Jacobiidentiät
\[
[u,[v,w]] + [v,[w,u]] + [w,[u,v]] = 0
\]
\end{enumerate}
\fl{Dann heißt $(V,[\cdot,\cdot])$ eine \texttt{Liealgebra}. Die Abbildung 
$[\cdot,\cdot]$ heißt die \texttt{Lieklammer} dieser Liealgebra.}
}
{}

\nonameyet
{Definition und Satz} {Liealgebra auf $\Gamma(TM)$}
{
\small
Es ist $\Gamma(TM)$ ein $\R$-Vektorraum. 
Weiter sei
\[
[\cdot,\cdot]: \Gamma(TM)\times \Gamma(TM) \to \Gamma(TM)
\]
gegeben durch 
\[
[X,Y] : M \to T_pM, \; p \mapsto [X,Y]_p := [X,Y](p) = X_pY - Y_pX.
\]
Es ist zu zeigen, dass dies wohldefiniert ist, dass also
\[
[X,Y]_p : \mathcal{F}_p \to \R, [f] \mapsto [X,Y]_p(f) = X_p(Y(f)) - Y_p(X(f))
\]
die Leibnisregel erfüllt.

\fl{Weiter gilt:}

$(\Gamma(TM),[\cdot,\cdot])$ ist eine reelle Lielalgebra.
}
{}
%%%%%%%%%%%%%%%%%%%%%%%%%%%%%%%%%%%%%%%%%%%%%%%%%%%%%%%%%%%%%%%%%%%%%%%%%%%%%%%
% 1.10 Integralkurven und Flüsse von Vektorfeldern
%%%%%%%%%%%%%%%%%%%%%%%%%%%%%%%%%%%%%%%%%%%%%%%%%%%%%%%%%%%%%%%%%%%%%%%%%%%%%%%
\kommentar{\tiny 1.10 Integralkurven und Flüsse von Vektorfeldern}

\nonameyet
{Definition} {Integralkurve}
{
Sei $X$ ein Vektorfeld auf M, $\alpha: (a,b) \to M$ eine differenzierbare
Kurve.
\\
\fl{Dann heißt $\alpha$ eine \texttt{Integralkurve von X}, falls} 
\[
\dot \alpha (t) = X_{\alpha(t)} \quad \forall t \in (a,b)
\]
}
{}

\nonameyet
{Rechnung} {ODE für Integralkurven}
{
Sei $(U,\varphi)$, $\varphi = (x_1,\dots,x_n)$ und 
$\alpha_i = x_i \circ \alpha$.
\\
\fl{Wir setzen} 
$$F_i: \varphi(U) \to \R, \quad F_i(\varphi(q)) = (Xx_i)(q) = X_q(x_i)$$
\fl{Dann gilt}
\small
\[
\alpha \text{ ist Integralkurve von } X 
\Leftrightarrow
\alpha_i'(t) = F_i (\alpha_1(t),\dots,\alpha_n(t)), \; i = 1,\dots,n
\]
}
{}

\nonameyet
{Satz} {Lsg ODE}
{
	\small
Sei $V\subset \R^n$ offen und $F: V\to \R$ differenzierbar. $I$ bezeichne Intervall.
\\
\normalsize
\fl{Dann gilt:}
\begin{enumerate}[1.]
\item \texttt{Existenz:}
\vspace{-1em} 
\[
\forall q\in V \; \exists I\ni 0,  
c\in C^\infty(I,V)\text{ mit } c(0) = q, c'(t) = F(c(t))  
\]
%
\item \texttt{Eindeutigkeit:} Gilt für $c_i \in C^\infty(I,V)$, $i = 1,2$
\vspace{-0.8em}
\[
c_i'(t) = F(c_i(t)) \quad (i =1,2)
\]
und
\[
\exists t_0 \in I: c_1(t_0) = c_2(t_0)
\]
Dann gilt $c_1 = c_2$.
\end{enumerate}
}
{}

\nonameyet
{Satz} {LSG ODE auf Mf}
{
\small
Sei $X$ ein Vektorfeld auf einer differenzierbaren Mf. $I,J$ seien Intervall.
\fl{Dann gilt:}
\begin{enumerate}[1.]
\item \texttt{Existenz:} Es gibt durch jeden $p\in M$ eine Integralkurve von $X$, d.h.: 
\[
\forall p\in M \; \exists I \ni 0, \alpha \in C^\infty(I,M) \text{ mit }
\alpha(0) = p, \alpha'(t) = X_{\alpha(t)}.
\]
%\vspace{-0.2em}
%
\item \texttt{Eindeutigkeit:} Sind $\alpha_i : I \to M$ mit $i = 1,2$ 
zwei Integralkurven von $X$ mit $\alpha_1(t_0) = \alpha_2(t_0)$ für 
ein $t_0 \in I$, dann gilt $\alpha_1 = \alpha_2$.
\end{enumerate}
\fl{Weiter folgt:}
\vspace{-1em}
Es gibt zu jedem $p\in M$ eine maximal definierte Integralkurve 
$\alpha \in C^\infty(I,M)$ mit $\alpha(0) = p$, d.h.
\[
\exists \beta \in C^\infty(J, M) \text{ mit } 
0\in J, \beta(0) = p, \beta'(t) = X_{\beta(t)},
\]
\fl{so gilt}
\[
J \subset I \text{ und } \alpha|_J = \beta
\]
}
{}

\nonameyet
{Satz 1.10.5} {}
{
\small 
Sei $M$ diffb. Mf, $X$ Vektorfeld auf $M$. 
\fl{Dann existiert eine Umgebung $U$ um $p \in M$, ein Intervall $I$ 
mit $0\in I$, sowie}
\[
\Phi\in \C^\infty(I\times U, M): 
\]
\fl{so dass gilt}
\begin{enumerate}[1.]
\item $\Phi(0,q) = q$ für alle $q\in U$.
\item $\alpha : I \to M$, $t \mapsto \Phi(t,q)$ ist eine Integralkurve von X,
d.h.
\[
\alpha'(t) = \frac{\partial}{\partial t} \Phi(t,q) = X_{\alpha(t) = \Phi(t,q)}
\]
\end{enumerate}
}
{}

\nonameyet
{Definition} {lokaler und globaler Fluss}
{
Sei $M$ diffb. Mf, $X$ Vektorfeld auf $M$. (Vor. Satz. 1.10.5) 
\fl{Dann heißt $\Phi\in C^\infty(I\times U, M)$ \texttt{lokaler Fluss}
von $X$.}
\fl{Falls $I = \R$ so heißt $\Phi$ \texttt{globaler Fluss}.}
\small
Es gilt weiterhin, dass wenn $t,s \in I$, $q\in U$, ist $\Phi(t,q) \in U$ und 
\[
\Phi(s, \Phi(t,q)) = \Phi(s + t, q)
\]
}
{}

\nonameyet
{Definition} {\small vollständiges Vektorfeld}
{
Sei $M$ diffb. Mf, $X$ Vektorfeld auf M.
\fl{Dann heißt $X$ \texttt{vollständig}, wenn durch jeden Punkt $p\in M$
eine Integralkurve läuft, die auf ganz $\R$ definiert ist, wenn also}
\[
\forall p\in M \; \exists \alpha \in C^\infty(\R,M) 
\text{ mit } \alpha(0) = p, \alpha'(t) = X_{\alpha(t)} \; \forall t\in\R
\]
erfüllt ist.
}
{}

\nonameyet
{Satz} {}
{
Sei $M$ diffb. Mf.
\\
~\\
Ist $M$ kompakt, so ist jedes Vektorfeld auf $M$ vollständig.
}
{}

\nonameyet
{Satz} {}
{
Sei $M$ diffb. Mf.
\\
~\\
Ist $X$ ein vollständiges Vektorfeld, dann existiert ein globaler Fluss
auf X.
}
{}

\nonameyet
{Definition} {\small Einparametergruppe von Diffeomorphismen}
{
\tiny
Sei $M$ diffb. Mf. $X$ ein vollständiges Vektorfeld. 
$\Phi: \R \times M \to M$ der globale Fluss von $X$. 
\scriptsize
%
\fl{Wir definieren:}
\vspace{-1em}
\[
\Phi_t : M \to M, \; p \mapsto \Phi_t(p) := \Phi(t,p)
\]
%
\fl{Dann ist}
\vspace{-1em}
\[
\Phi_{t+s}(p) = \Phi(t+s,p) = \Phi(t,\Phi(s,p)) = (\Phi_t \circ \Phi_s)(p)
\]
%
\fl{also}
\vspace{-1em}
\[
\Phi_{t+s} = \Phi_t \circ \Phi_s 
,\quad 
\Phi_{-t} = \Phi^{-1}_t
, \quad 
\Phi_0 = id_M
\]
\fl{Damit definieren wir}
%
\small
Sei $\Psi \in C^\infty(\R \times M, M)$ und gilt für 
$\Psi_t: t\mapsto \Psi(t,p)$, 
\[
\Psi_0 = id_M 
\quad \text{und} \quad
\Psi_{t+s} = \Psi_t\circ \Psi_s \; \forall t,s \in \R
\]
%
so heißt $\Psi$ \texttt{Einparametergruppe von 
Diffeomorphismen}.
}
{}

\nonameyet
{\tiny Bemerkung} {\tiny vollständiges Vektorfeld und Einparametergruppen}
{
Hier ausführen wie vollständige Vektorfelder und Einparametergruppen
einandern zugeordnet werden können.
\\
TODO
}
{}

%%%%%%%%%%%%%%%%%%%%%%%%%%%%%%%%%%%%%%%%%%%%%%%%%%%%%%%%%%%%%%%%%%%%%%%%%%%%%%%%
% 2.1 (Pseudo-) Riemannsche Mannigfaltigkeit
%%%%%%%%%%%%%%%%%%%%%%%%%%%%%%%%%%%%%%%%%%%%%%%%%%%%%%%%%%%%%%%%%%%%%%%%%%%%%%%%
\kommentar{\scriptsize 2.1 (Pseudo-) Riem. Mf}

\nonameyet
{Definition 1} {nicht-entartet}
{
	\small
Es sei $V$ ein endlich-dim. $\R$-Vektorraum, $\blf{\cdot,\cdot}$ eine 
symmetrische Bilinearform auf $V$.
\fl{Dann heißt $\blf{\cdot,\cdot}$ \texttt{nichtentartet}, falls}
\[
\forall v\in V\setminus\{0\} \; \exists w\in V: \blf{v,w} \neq 0.
\]
\fl{Oder gleich bedeutend damit:}
\begin{itemize}[-]
\item%
Die lineare Abbildung $V \to V^*, v \mapsto \blf{v,\cdot}$ ist injektiv, d.h.
\[
\blf{v,w} = 0 \quad \forall w\in V \Rightarrow v = 0
\] 
\item%
Für (beliebige) $v_1,\dots,v_n$ Basis von $V$ gilt
\[
\det(g_{ij}) \neq 0 \quad \text{ mit } g_{ij} = \blf{v_i,v_j} 
\]
\end{itemize}
}
{}

\nonameyet
{Definition 2} {Signatur,\dots}
{
	\small
Sei $V$ ein endlich-dim. $\R$-VR.
\\
Und $\blf{\cdot,\cdot}$ sei eine nichtentartete symmetrische Bilinearform. 
\begin{itemize}[- ]
\item Eine nichtentartete symmetrische Bilinearform $\blf{\cdot,\cdot}$
auf $V$ heißt auch \texttt{pseudo-Euklidisches Skalarprodukt}.
\item $(V,\blf{\cdot,\cdot})$ heißt \texttt{pseudo-Euklidscher Vektorraum}.
\item $(V,\blf{\cdot,\cdot})$ sei ein $n$-dim. pseudo-Euklidscher Vektorraum 
und $p$ die maximale Dimension eines Unterraums, auf dem $\blf{\cdot,\cdot}$ 
positiv-definit ist, so heißt \texttt{$(n-p,n)$ die Signatur von $V$}.
\end{itemize}
Sei $\blf{\cdot,\cdot}$ zusätzlich positiv-definit, so ist die Signatur $(0,n)$ 
und
\begin{itemize}[-]
\item $\blf{\cdot,\cdot}$ heißt dann \texttt{(Euklidsches) Skalarprodukt}. 
\item $(V,\blf{\cdot,\cdot})$ heißt ein \texttt{Euklidscher Vektorraum}.
\end{itemize}
}
{}

\nonameyet
{\small Definition 3 und Notiz} {\small pseudo-Riemmansche Metrik}
{
\footnotesize
Sei $M$ eine diffb. Mf, $g\in \Gamma(T^0_2(M))$.
\fl{Falls gilt}
$g(p) : T_pM \times T_pM \to \R$  ist für alle $p\in M$
ein pseudo-Euklidisches Skalarprodukt auf $T_pM$,
\fl{so heißt, das symmetrische $(0,2)$-Tensorfeld $g$ eine 
\texttt{pseudo-Riemannsche Metrik}.}
\fl{Das Paar $(M,g)$ heißt \texttt{pseudo-Riemannsche Mannigfaltigkeit}.}
Sei nun $g_p := g(p)$ für alle $p\in M$ positiv-definit,
\fl{dann heißt $g$ eine \texttt{Riemannsche Metrik} und $(M,g)$ eine 
\texttt{Riemannsche Mannigfaltigkeit}.}
Sei $(M,g)$ eine zusammenhängende pseudo-Riemannsche Mf, so ist 
die Signatur aus stetigkeitsgründen von $g_p$ konstant auf $M$.
\fl{Allgemein nennt man eine konstante Signatur, die 
\texttt{Signatur von $M$}.}
}
{}

\nonameyet
{Definition 3} {Lorentz-Mannigfaltigkeit}
{
Sei $(M,g)$ eine pseudo-Riemannsche Mf der Signatur (1,p).
\fl{Dann heißt $(M,g)$ eine \texttt{Lorentz-Mannigfaltigkeit}}
}
{}

\nonameyet
{Beispiele} {}
{
TODO, per Hand
\\
insbesondere zu Untermannigfaltigkeiten und Produktmannigfaltigkeiten
\\
Auch wie man $(g_{ij})$ als Matrix auffassen kann und Satz von Sylvester
}
{}

\nonameyet
{Satz 1 und Notiz} {Existenz Riem. Metrik}
{
Auf jeder differenzierbaren Mannigfaltigkeiten $M$ existiert eine Riemannsche 
Metrik.
\\
\fl{Notiz: Diese Aussage gilt nicht für pseudo-Riemannsche Mf.}

}
{}
%%%%%%%%%%%%%%%%%%%%%%%%%%%%%%%%%%%%%%%%%%%%%%%%%%%%%%%%%%%%%%%%%%%%%%%%%%%%%%%%
% 2.3 Zusammenhänge
%%%%%%%%%%%%%%%%%%%%%%%%%%%%%%%%%%%%%%%%%%%%%%%%%%%%%%%%%%%%%%%%%%%%%%%%%%%%%%%%
\kommentar{2.3 Zusammenhänge}


\nonameyet
{Definition \thedef} {(affiner) Zusammenhang}
{
	\small
Sei $M$ eine differenzierbare Mf.
\\
Weiter sei eine Abbildung gegeben durch
\[
\nabla: \Gamma(TM) \times \Gamma(TM) \to \Gamma(TM), 
\quad (X,Y) \mapsto \nabla_X Y,
\]
\fl{die folgende Eigenschaften für alle $X,Y,Z\in \Gamma(TM)$ und 
für alle $f,g \in C^\infty(M)$ erfüllt:}
\begin{enumerate}[1.]
\item 
$\nabla_{fX+gY}Z = f\nabla_X Z + g \nabla_Y Z$ 
\hfill (\texttt{$C^\infty$-linear in 1. Komponente})
\item $\nabla_X(Y+Z) = \nabla_X Y + \nabla_YZ$
\hfill ( \texttt{additiv in 2. Komponente})

$\nabla_X(fY) = f\nabla_X Y + X(f)Y$
\hfill ( \texttt{``Produktregel'' in 2. Komponente})
\end{enumerate}

\fl{Dann heißt $\nabla$ eine \texttt{(affinier) Zusammenhang} oder 
\texttt{kovariante Ableitung}.}
}
{}

\nonameyet
{Satz 1 und Folgerung} {Lokalität}
{
\small
Sei $M$ differenzierbare Mf, $p\in M$ und $\nabla$ eine affinier Zusammenhang.
\fl{Dann gilt für alle $X_1,X_2,Y \in \Gamma(TM)$:}
Falls $X_1(p) = X_2(p)$, so folgt
$
(\nabla_{X_1} Y)(p) = (\nabla_{X_2} Y)(p)
$

\fl{ \textsc{Folgerung:} 
$v\in T_pM$ kann zu einem beliebigen Vektorfeld $X\in\Gamma{(TM)}$
fortgesetzt werden. Sei $Y\in\Gamma(TM)$ so setzen wir:}
\[
(\nabla_v Y) := (\nabla_XY)(p).
\]
\fl{Die Wohldefiniertheit dieser Abbildung, d.h. dass sie unabhängig
von der Wahl der Erweiterung ist, folgt gerade aus Satz 1.}
}
{}

\nonameyet
{Satz 2}{Verallgemeinerung von Satz 1}
{
\small
Sei $M$ diffb. Mf und 
$A: \overbrace{\Gamma(TM)\times \dots \times \Gamma(TM)}^{s\text{-mal}} 
\to \Gamma(TM)$ eine $s$-multilineare Abbildung für die gilt,
{dass für alle $f\in C^\infty(M)$ und 
für alle $X_1,\dots,X_s \in \Gamma(TM)$:}
\[
A(X_1,\dots,fX_i,\dots,X_s) = fA(X_1,\dots,X_s)
\quad \forall i \in \{1,\dots,s\}.
\]
\fl{Dann existiert ein $(1,s)$-Tensorfeld $B \in \Gamma(T^1_s(M))$ auf $M$,
so dass }
\[
A(X_1,\dots,X_s)(p) = B_p(X_1(p),\dots,X_s(p)) 
\quad \forall X_1,\dots, X_s \in \Gamma(TM), \forall p\in M.
\]
%
\fl{$\point$ Eine analoge Aussage gilt für $(r,s)$-Tensorfelder. 
Hier betrachtet man multilineare Abbildung}
\[
A: \underbrace{\Gamma(T^*M)\times \dots \times \Gamma(T^*M)}_{r\text{-mal}} 
\times 
\underbrace{\Gamma(TM) \times \dots \times \Gamma(TM)}_{s\text{-mal}} 
\to C^\infty(M),
\]
die $C^\infty(M)$-linear in jedem Eintrag sind.
}
{}

\nonameyet
{Bsp und Bem.} {}
{
TODO Seite 36
\\
$X\mapsto \nabla_X Y$ ist eine $(1,1)$-Tensorfeld.
}
{}

\nonameyet
{Satz 3} {}
{
Sei $M$ eine diffb. Mf mit affinen Zusammenhang $\nabla$, $p\in M$, $v\in T_pM$,
$Y_1,Y_2 \in \Gamma(TM)$.
\\
~\\
Gilt, dass  $Y_1$ und $Y_2$ in einer Umgebung übereinstimmen, so folgt 
$\nabla_v Y_1 = \nabla_v Y_2$.
}
{}

\nonameyet
{Definition und Notiz} {Christoffel-Symbole}
{
\small
Sei $M$ diffb. Mf, $(U,\varphi)$, $\varphi = (x_1,\dots,x_n)$ lokale
Koordinaten auf $M$.

\fl{Als Konsequenz der Sätze dieses Abschnitts ist}
\[
\nabla_{\pd} \pd[x_j] : U \to TU
\]
\fl{wohldefiniert.}
%
\fl{Als Element in $\Gamma(TU)$ können wir es in eine Basis schreiben }
\[
\nabla_{\pd} \pd[x_j] = \sum_{k=1}^n \Gamma^k_{ij} \pd[x_k]
\]
\fl{und definieren darüber die \texttt{Christoffel-Symbole} 
$\Gamma^k_{ij} \in C^\infty(U)$, welche den Zusammenhang $\nabla$ auf 
$U$ bestimmen.}
}
{}
%%%%%%%%%%%%%%%%%%%%%%%%%%%%%%%%%%%%%%%%%%%%%%%%%%%%%%%%%%%%%%%%%%%%%%%%%%%%%%%%
% 2.4 Vektorfelder längs Kurven
\defreset
\satzreset
%%%%%%%%%%%%%%%%%%%%%%%%%%%%%%%%%%%%%%%%%%%%%%%%%%%%%%%%%%%%%%%%%%%%%%%%%%%%%%%%
\kommentar{\scriptsize 2.4 Vektorfelder längs Kurven}

\nonameyet
{\small Definition \thisdef und Notizen} {\small Vektorfeld längs $c$}
{
\small
Sei $M$ diffb. Mf, $I = [a,b] \subset \R$, $c\in C^\infty(I,M)$
\fl{Eine differenzierbare Abbildung  }
\[
X: I \to TM,\; t \mapsto X_t := X(t) \text{ mit } X_t \in T_{c(t)}M \; \forall t\in I
\]
\fl{heißt \texttt{Vektorfeld längs $c$}.}
\fl{Wir bezeichnen den Raum der Vektorfelder längs $c$ mit $\Gamma_c(TM)$.}

\fl{\textsc{Notiz:} $\dot c$ ist ein Vektorfeld längs $c$. \\
Genauso wie $X\circ c$ mit $X\in\Gamma(TM)$, $c\in C^\infty(I,M)$.
Andersherum ist nicht jedes Vektorfeld längs einer Kurve Einschränkung
eines Vektorfeldes auf $M$. Diese Aussage sollte jedoch lokal gelten.
}
\fl{\textsc{Notiz:} $\Gamma_c(M)$ ist ein Modul über $C^\infty(I)$.}
\fl{\textsc{Notiz:} Für $X \in \Gamma(TM)$ ist 
$(\nabla_{\dot c} X)(t) := \nabla_{\dot c(t)} X \in \Gamma_c(TM)$.}
}
{}


\nonameyet
{\scriptsize Definition \thisdef \& Satz \thissatz} 
{\scriptsize Kovariante Ableitung längs $c$}
{
	\scriptsize
	\vspace{-2em}
Sei $M$ diffb. Mf mit affinem Zusammenhang $\nabla$, $I =[a,b] \in \R$, 
$c\in C^\infty(I,M)$.
\fl{Dann existiert eine eindeutige Abbildung}
\[
\nabladt: \Gamma_c(TM) \to \Gamma_c(TM), \quad X \mapsto \nabladt X,
\]
\fl{die folgende drei Eigenschaften erfüllt für alle $X,Y \in \Gamma_c(TM)$,
$f\in C^\infty(I)$:}
\begin{enumerate}[1.]
\item $\nabladt(X+Y) = \nabladt X + \nabladt Y$
\item $\nabladt(fX) = f' X + f \nabladt X$,
\end{enumerate}
\fl{ist $X = Z \circ c$ für ein $Z \in \Gamma(TM)$, so gilt weiter}
\begin{enumerate}[3.]
\item $\nabladt X = \nabla_{\dot c} Z$.
\end{enumerate}
\fl{$\nabladt$ heißt auch \texttt{kovariante Ableitung längs $c$ bzw. einer
Kurve}.}
\fl{Weiter gilt in lokalen Koordinaten $(U,\varphi)$, 
\vspace{-1em}
$\varphi = (x_1,\dots,x_n)$ und $f_i \in C^\infty(c^{-1} (U))$:}
\[
\text{Für } X = \sum_{i=1}^n f_i \cdot \pd[x_i] \circ c 
\quad \text{ ist } \quad 
\nabladt X = \sum_k \left ( f_k' + \sum_{ij} (x_i \circ c)' f_j 
\Gamma^k_{ij} \circ c \right) \cdot \pd[x_k] \circ c
\]
}
{}
%%%%%%%%%%%%%%%%%%%%%%%%%%%%%%%%%%%%%%%%%%%%%%%%%%%%%%%%%%%%%%%%%%%%%%%%%%%%%%%%
% 2.5 Parallelverschiebung
%%%%%%%%%%%%%%%%%%%%%%%%%%%%%%%%%%%%%%%%%%%%%%%%%%%%%%%%%%%%%%%%%%%%%%%%%%%%%%%%
\defreset
\satzreset
\kommentar{\small 2.5 Parallelverschiebung}

\nonameyet
{Definition \thisdef} {paralleles VF}
{
Sei $M$ diffb. Mf, $I = [a,b] \subset \R$, $c\in C^\infty(I,M)$ und 
$X \in \Gamma_c(TM)$.
\fl{Falls $\nabladt X = 0 $ gilt, so heißt das Vektorfeld $X$ längs $c$ 
\texttt{parallel}.}
}
{}

\nonameyet
{Satz \thissatz und Notiz} {Existenz VF längs $c$}
{
Sei $M$ diffb. Mf mit affinen Zusammenhang $\nabla$, 
$I = [a,b] \subset \R$, $c\in C^\infty(I,M)$ und sei $v \in T_{c(a)}M$.
\fl{Dann existiert ein eindeutiges Vektorfeld $X$ längs c,
so dass $X(a) = v$.}
\[
 \exists ! X \in \Gamma_c(TM): X(a) = v
 \quad \text{AWP}
\]
\fl{\textsc{Notiz:} $\Gamma_c(TM)$ ist ein Vektorraum und dieser Satz zeigt,
eine Basis dieses Vektorraums zu jedem $t\in I = [a,b]$ eine Basis
des Tangentialraums $T_{c(t)}M$ liefert.}
}
{}

\nonameyet
{Definition \thisdef \& Satz \thissatz} {\footnotesize Parallelverschiebung}
{
Sei $M$ diffb. Mf mit affinen Zusammenhang $\nabla$, 
$I = [a,b] \subset \R$, $c\in C^\infty(I,M)$ und sei 
\\ 
$X\in \Gamma_c(TM)$ das eindeutige Vektorfeld mit $X(a) = v\in T_{c(a)}$.
\fl{Die Abbildung}
\[
c||_a^b : T_{c(a)} \to T_{c(b)}, \quad v=X(a) \mapsto X(b) 
\]
heißt \texttt{Parallelverschiebung längs $c$ von $a$ nach $b$.}
\fl{\textsc{Satz:} $c||_a^b$ ist ein linearer Isomorphismus.}
}
{}

\nonameyet
{Satz \thissatz} {Differenzenquotient}
{
Sei $M$ diffb. Mf mit affinen Zusammenhang $\nabla$, 
$I = [-\eps,\eps]$ wobei $\eps >0$ und $c\in C^\infty(I,M)$ mit 
$c(0) = p \in M$, $\dot c(0) = v$.
\fl{Dann gilt}
\[
\nabla_v X = \lim_{t\to 0} \frac{c||_t^0 X_{c(t)} - X_p}{t}
\]
}
{}
%%%%%%%%%%%%%%%%%%%%%%%%%%%%%%%%%%%%%%%%%%%%%%%%%%%%%%%%%%%%%%%%%%%%%%%%%%%%%%%%
% 2.6 Levi-Civia-Zusammenhang
%%%%%%%%%%%%%%%%%%%%%%%%%%%%%%%%%%%%%%%%%%%%%%%%%%%%%%%%%%%%%%%%%%%%%%%%%%%%%%%%
\defreset
\satzreset
\kommentar{\footnotesize 2.6 Levi-Civita-Zusammenhang}

\nonameyet
{Satz \thissatz \& Definition \thisdef} {Torsionstensor}
{
	\small
Sei $M$ diffb. Mf mit affiner Zusammenhang $\nabla$.
\fl{Die folgende Abbildung $T$ ist ein $(1,2)$-Tensorfeld:}
\[
T : \Gamma(TM) \times \Gamma(TM) \to \Gamma(TM), 
\quad (X,Y) \mapsto T(X,Y) = \nabla_X Y - \nabla_Y X - [X,Y].
\]
\fl{$T$ heißt \texttt{Torison} oder \texttt{Torsionstensor}.}
\fl{Falls $T = 0$, so heißt der Zusammenhang \texttt{torisionsfrei}.}
}
{}

\nonameyet
{Definition \thisdef} {metrisch}
{
Es sei $(M,\blf{\cdot,\cdot})$ eine pseudo-Riem. Mf und $\nabla$ ein
Zusammenhang auf $M$.
\fl{Falls für alle $X,Y,Z \in \Gamma(TM)$ gilt, dass}
\[
X \blf{Y,Z} = \blf{\nabla_X Y, Z} + \blf{Y, \nabla_X Z},
\]
\fl{so heißt der Zusammenhang $\nabla$ \texttt{metrisch}.}
}
{}

\nonameyet
{Satz \thissatz} {}
{
\footnotesize
Sei $(M,\blf{\cdot,\cdot})$ eine pseudo-Riem. Mf und $\nabla$ ein 
affiner Zusammenhang von $M$. $I = [a,b] \subset \R$.
\fl{Dann gilt}
\[
\nabla \text{ ist metrisch}
\quad \Leftrightarrow \quad 
\frac{d}{dt} \blf{X,Y} = \blf{\nabladt X, Y} + \blf{X,\nabladt Y}
\;\; \forall c\in C^\infty(I,M),
\; \forall X,Y \in \Gamma_c(TM)
\]
}
{}

\newcommand{\dg}[3]{#1\blf{#2,#3}}
\newcommand{\blfL}[3]{\blf{#1,[#2,#3]}}
\nonameyet
{Satz \thissatz und Definition \thisdef} {Koszul-Formel}
{
Sei $(M,\blf{\cdot,\cdot})$ pseudo-Riem. Mf.
\fl{Dann exisiert genau ein torsionsfreier und metrischer Zusammenhang $\nabla$
auf $M$.}
\fl{Dieser ist durch die \texttt{Koszul-Formel}}
\[
\begin{aligned}
2\blf{\nabla_X Y, Z} =& 
\dg{X}{Y}{Z} + \dg{Y}{Z}{X} - \dg{Z}{X}{Y}
\\&
+ \blfL{Z}{X}{Y} + \blfL{Y}{Z}{X} - \blfL{X}{Y}{Z}
\end{aligned}
\]
\fl{bestimmt.}
\fl{\textsc{Definition:} Dieser eindeutiger Zusammenhang heißt 
\texttt{Levi-Civita-Zusammenhang}.}
}
{}
%%%%%%%%%%%%%%%%%%%%%%%%%%%%%%%%%%%%%%%%%%%%%%%%%%%%%%%%%%%%%%%%%%%%%%%%%%%%%%%%
% 2.7 Geodätische
%%%%%%%%%%%%%%%%%%%%%%%%%%%%%%%%%%%%%%%%%%%%%%%%%%%%%%%%%%%%%%%%%%%%%%%%%%%%%%%%
\kommentar{2.7 Geodätische}

\nonameyet
{Definition} {Geodätische}
{
\scriptsize
Sei $M$ eine diffb. Mf mit einem affinen Zusammenhang $\nabla$.
Lokale Koordinaten: $(U,\varphi)$, $\varphi = (x_1,\dots,x_n)$.
%
\\
~\\
\small
Sei $\gamma : I \to M$ eine Kurve, $\dot \gamma $ ist dann ein Vektorfeld
längs $\gamma$. Diese $\dot \gamma$ sei parallel, es gilt also
\[
\frac{\nabla}{dt} \dot\gamma = 0.
\]
\fl{Dann heißt $\gamma$ \texttt{Geodätische}.}
\footnotesize
\fl{Wir setzen $\gamma_i = x_i \circ \gamma$}
\[
\frac{\nabla}{dt} \dot \gamma 
= \sum_k \left( 
\gamma''_k + \sum_{i,j} \gamma'_i \gamma'_j \Gamma^k_{ij}\circ \gamma
\right) 
\cdot \partial_k \circ \gamma
\]
\fl{Also}
\[
\gamma|_U \text{ ist \texttt{Geodätische} }
\quad \Leftrightarrow \quad
\gamma''_k + \sum_{i,j} \gamma'_i \gamma'_j \Gamma^k_{ij}\circ \gamma = 0
\;\; \forall k
\]
}
{}

\nonameyet
{Satz} {\tiny Existenz und Eindeutigkeit der Geodätischen}
{
\small
Sei $M$ diffb Mf mit affinen Zusammenhang $\nabla$. 
\fl{Dann gilt:}
\begin{enumerate}[1.]
\item \texttt{Existenz} einer Geodäditschen $\gamma$: 
\vspace{-0.7em}
\[
\begin{aligned}
\forall p\in M, v\in T_pM \; \exists \epsilon > 0, \;
\gamma:(-\epsilon,\epsilon) \to M \text{ mit } 
\\ 
\gamma(0) = p, 
\quad 
\gamma'(0) = v 
\quad \text{und} \quad 
\frac{\nabla}{dt} \dot \gamma = 0
\end{aligned}
\]
%
\item \texttt{Eindeutigkeit:} Jede weitere Geodätische 
$\eta: (-\delta,\delta) \to M$ mit denselben Anfangsbedingungen
$\eta(0) = p$ und $\eta'(0) = v$ stimmt auf einem Intervall um $0$
mit $\gamma$ überein.
\end{enumerate}
%
\scriptsize
Unter Missachtung des Definitionsbereiches, sagt man, die Geodätische 
$\gamma$ mit $\gamma(0) = p$, $\gamma'(0) = v$ ist eindeutig und schreibt
$\gamma_v$.
\\
}
{
	TODO 
	\\
	Diese Konstruktion gerne texen
}

\nonameyet
{Definition} {geodätischer Fluss}
{
TODO : Konstruktion S.46 
%
\\
Der lokale Fluss von $Y$ heißt geodätsicher Fluss.
\scriptsize 
\fl{Dies ist so zu verstehen:}
$Y$ ist Vektorfeld auf $TTU$. Eine Integralkurve zu $Y$ heiße
$\gamma : I \to TU$. Dann ist der lokale Fluss eine Abbildung
$\Phi: I \times TU \to TTU \overset{\sim}{=} TU$ mit 
$ \Phi(t,v) = \dot\gamma_v(t)$ für alle $t\in I$, die noch zusätzliche 
Bedingung erfüllt.
}
{}

\nonameyet
{Satz} {}
{
Sei $M$ diffb Mf mit Zusammenhang $\nabla$. Sei $\Phi$ der geodätische
Fluss um die Punkt $0 = 0_p \in T_p M$, so gilt:
\scriptsize
\[
\forall p\in M \; \exists 0_p \in V \overset{\text{off}}{\subset} TM,
\delta > 0, \Phi\in C^\infty((-\delta,\delta)\times V,TM)
: \exists! \Phi_v : t\mapsto \Phi(t,v), 
\]
wobei $\Phi_v$ Integralkurve von $Y$ ist.
}
{}

%%%%%%%%%%%%%%%%%%%%%%%%%%%%%%%%%%%%%%%%%%%%%%%%%%%%%%%%%%%%%%%%%%%%%%%%%%%%%%%%
% 2.8. Riemannsche Mannigfaltigkeiten als metrsiche Räume
%%%%%%%%%%%%%%%%%%%%%%%%%%%%%%%%%%%%%%%%%%%%%%%%%%%%%%%%%%%%%%%%%%%%%%%%%%%%%%%%
\kommentar{\tiny 2.8 Riem MF als metrische Raum}
\nonameyet
{\tiny Definition} {\tiny Länge einer stückweise diffb. Kurve, Abstand}
{
\scriptsize
Es $c: [a,b] \to M$ eine stetige Abbildung. 
Existiert eine Unterteilung $a = t_0 < t_1 < \dots < t_k = b$, so dass
$c|_{[t_i,t_{i+1}]} \in C^\infty$ für alle $i = 0,\dots, k-1$ ist,
so heißt $c$ \texttt{stückweise differenzierbare Kurve}.
\\
~\\
Sei $(M,g)$ eine Riemannsche Mf und $c : [a,b] \to M$ eine stückweise
differenzierbare Kurve, so definieren wir durch
\[
L(c) = \int_a^b \| \dot c(t) \| dt,
\]
die \texttt{Länge} der Kurve $c$. 
Die Länge einer stückweise differenzierbaren Kurve ist endlich.
%\\
%~\\
\fl{Weitere Definitionen:}
\begin{itemize}[-]
\item Für $p,q \in M$ sei $\Omega_{pq}$ die Menge aller stückweise
			diffb. Kurven in $M$ von $p$ nach $q$.  
\item %
Eine monotone, surjektive Abbildung $\varphi \in C^\infty([c,d],[a,b])$ heißt
		\texttt{differenzierbare Umparametrisierung}.
\item %
Eine Kurve heißt zur Bogenlänge parametrisiert, falls
$\| \dot c \|  = 1$ ist und (proportional) zur Bogenlänge parametrisiert,
falls $\| \dot c \|$ konstant ist
\end{itemize}
}
{}

\nonameyet
{Satz} {}
{
	\small
\fl{Es gelten folgende Aussagen:}
\begin{enumerate}[1.]
\item 
\begin{itemize}
\item $L(c) \geq 0$; 
\item $L(c) = 0$ genau dann, wenn $c$ konstant ist.
\end{itemize}
%
\item Sind $c_1 : [a,b] \to M$, $c_2 : [b,c] \to  M$ zwei stückweise
differenzierbare Kurven mit $c_1(b) = c_2(b)$, so ist 
$$L(c_1 \cup c_2) = L(c_1) + L(c_2),$$ 
wobei $c_1 \cup c_2$ die Konkatenation von $c_1$ und $c_2$ bezeichnet.
\item Es sei $\varphi : [c,d] \to [a,b]$ eine differenzierbare 
Umparametrisierung, 
\\ $c : [a,b] \to M$ ein beliebiger stückweise 
differenzierbarer Weg.
%\\
Dann gilt  $L (c \circ \varphi) = L (c)$.
\end{enumerate}
}
{}

\nonameyet
{Satz} {}
{
\fl{Es sei:} 
\vspace{-1.1em}
\[
d(p,q) = \inf \{ L(c) | c \in \Omega_{pq} \}.
\]
\fl{Dann gilt}
\begin{enumerate}[1.]
\item $(M,d)$ ist ein metrischer Raum. 
\item Die durch $d$ induzierte Topologie auf $M$ stimmt mit der ürsprünglichen
Topologie von $M$ als Mannigfaltikeit überein.
\end{enumerate}
}
{}

%%%%%%%%%%%%%%%%%%%%%%%%%%%%%%%%%%%%%%%%%%%%%%%%%%%%%%%%%%%%%%%%%%%%%%%%%%%%%%%%
% 2.9 Geodätische minimieren die Länge
%%%%%%%%%%%%%%%%%%%%%%%%%%%%%%%%%%%%%%%%%%%%%%%%%%%%%%%%%%%%%%%%%%%%%%%%%%%%%%%%
\kommentar{\scriptsize 2.9 Geodätische minieren die Länge}
\nonameyet
{Hauptsatz 1} {\scriptsize 2 Lemma, 1 Bem.}
{
Sei $\gamma : [0,1] \to M$ eine stückweise differenzierbare, proportional
zur Bogenlänge parametrisierte Kurve mit $\gamma(0) = :p$, $\gamma (1) = :q$
und gilt 
\[
L(\gamma)  \leq L(c) \quad \forall c \in \Omega_{pq}. 
\]
Dann ist $\gamma$ Geodätische.
}
{}

\nonameyet
{Definition} {\tiny normale Umgebung, geodätischen Ball}
{
\small
Sei $(M,g)$ (pseudo)-Riemannsche Mf, $p\in M$. 
Sei $V \subset T_p M$ offene Umgebung von $0 \in T_p M$.
\\
~\\
\fl{Ist $\exp_p$ auf $V$ ein
Diffeomorphismus aufs Bild, so heißt
$\exp_p(V)$ eine \texttt{normale Umgebung von $p$}.}
%\\
\fl{Ist $(M,g)$ Riemannsche Mf und ist $\epsilon > 0$ so, dass
$B_\epsilon(0) \subset V$. 
Dann heißt $\exp_p(B_\epsilon(0))$ ein \texttt{geodätsicher Ball um $p$.}}
}
{}

\nonameyet
{Hauptsatz 2} {\scriptsize 2 Lemma, 1 Kor.}
{
\small
Sei $p\in M$, $U$ eine normale Umgebung von $p$, $B \subset U$ ein 
geodätischer Ball um $p$, $\gamma:[0,1] \to M$ eine Geodätische mit 
$\gamma(0) = p$, die ganz in $B$ verläuft und $\gamma(1) = p'$.
%
\fl{Dann gilt}  
\[
L(\gamma) \leq L(c) \quad \forall c \in \Omega_{pp'}
\quad \text { \scriptsize insbesondere:} \quad 
L(\gamma) = d(\gamma(0),\gamma(1))
\]
\fl{Falls $L(\gamma) = L(c)$, dann gilt }
\begin{itemize}[-]
\item $\gamma([0,1]) = c([0,1])$ und $c$ ist Umparametrisierung von $\gamma$.
\item Insbesondere: Für jeden Punkt $q \in B$ gibt es bis auf Umparametrisierung
genau eine minimierende Geodätische, die $p$ mit $q$ verbindet.
\end{itemize}
Bemerkung: die Parametrisierung auf $[0,1]$ ist nicht relevant.
}
{}

\nonameyet
{Lemma 1} {zu HS 2}
{
Es sei $f: (a,b) \times (c,d) \to M$, $(t,s) \mapsto f(t,s)$ eine 
differenzierbare Abbildung.
\fl{Dann gilt:}
\[
\frac{\nabla}{ds} \frac{\partial f}{\partial t} 
= \frac{\nabla}{dt} \frac{\partial f}{\partial s}
\]	
TODO: Vielleicht ein Wort dazu wie diese Ableitung zu verstehen sind
}
{}

\nonameyet
{Lemma 2} {\small zu HS 2, Gauß-Lemma}
{
Es sei $p\in M$, $v\in T_pM$ so, dass $\exp_p v$ definiert ist
und $w\in T_v (T_p M) \overset{\sim}{=} T_pM$. 

\fl{Dann gilt:}
\[
\langle (d \exp_p)_v (v), (d \exp_p)_v(w) \rangle
= \langle v, w \rangle 
\]
}
{}

\nonameyet
{Korollar} {zu HS 2}
{
Voraussetzung weiter wie in HS2. Sei weiter 
$\gamma(t) = \exp_p(tw)$.
\fl{Dann gilt:}
\[
d(p,\gamma(t)) = \| tw \| = L(\gamma|_{[0,t]}).
\]
Insbesondere: Betrachten wir $M$ als metrischen Raum, so gilt für den
geodätischen Ball $B = \exp_p(B_\eps (0))$, dass $B = B_\eps(p)$.
}
{}


\nonameyet
{Lemma 1} {zu HS 1}
{
Es sei 
\[
F: TM \to M \times M, \quad v \mapsto (\pi(v), \exp(v)).
\]
\fl{Dann ist}
\[
\forall p\in M : dF_{0_p}: T_{0_p}TM \to T_{(p,p)}(M \times M)
\overset{\sim}{=} T_pM \oplus T_pM
\]
ist eine Isomorphismus.
}
{}


\nonameyet
{Lemma 2 + Bem.} {zu HS 1}
{
%Lokal(?), genauer:
\small
\[
\forall p \in M \; \exists \; \text{offene Umgebung $U$ von } p, \eps > 0: 
\forall q\in U
\]
folgendes gilt:
\begin{enumerate}[1.]
\item  Die Abbildung $\exp_q$, eingeschränkt auf $B_\eps(0) \subset T_qM$ 
ist ein Diffeomorphismus aufs Bild.
\item $U\subset \exp_q(B_\eps(0)) = B_\eps(q)$.
\end{enumerate}
%
\fl{Dies bedeutet:}
$U$ ist eine normale Umgebung eines jeden Punktes.
\fl{\texttt{Bemerkung:}
Aus diesem Lemma und HS 2, folgt, dass für je zwei Punkte $q_1, q_2 \in U$ 
bis auf Umparametrisierung eine eindeutige minimierende Geodätische 
$\gamma$ (mit Länge $< \eps$) existiert, die $q_1$ und $q_2$
miteinander verbindet.
\\
Man bezeichnet eine solche Umgebung als geodätisch konvex.}
}
{}

\nonameyet
{Bem} {zu HS 1}
{
Aus Lemma 2 und HS 2, folgt, dass für je zwei Punkte $q_1, q_2 \in U$ bis auf
Umparametrisierung eine eindeutige minimierende Geodätische $\gamma$ (mit 
Länge $< \eps$) existiert, die $q_1$ und $q_2$ miteinander verbindet.
\\
Man bezeichnet eine solche Umgebung als geodätisch konvex.
}
{}

%%%%%%%%%%%%%%%%%%%%%%%%%%%%%%%%%%%%%%%%%%%%%%%%%%%%%%%%%%%%%%%%%%%%%%%%%%%%%%%%
% 2.10 Satz von Hopf-Rinow 
%%%%%%%%%%%%%%%%%%%%%%%%%%%%%%%%%%%%%%%%%%%%%%%%%%%%%%%%%%%%%%%%%%%%%%%%%%%%%%%%
\defreset
\satzreset
\kommentar{2.10 Satz von Hopf-Rinow}


\nonameyet
{Definition \thisdef} {\small (geodätisch) vollständig}
{
	\small
Sei $(M,g)$ eine pseudo-Riem. Mf mit Levi-Civita-Zusammenhang $\nabla$.
\fl{Gilt, dass für alle $p\in M$ die Exponentialabbildung} 
\[
\exp_p : T_pM \to M
\]
\fl{auf ganz $T_pM$ definiert ist, d.h. jede Geodätische von $M$ kann auf 
ganz $\R$ erweitert werden, dann nnen wir $(M,g)$ 
\texttt{(geodätisch) vollständig}.}
}
{}

\nonameyet
{\small Satz und Bemerkungen} {\small Satz von Hopf-Rinow}
{
	\vspace{-2em}
\scriptsize
Es sei $(M,g)$ zusammenhängende Riem. Mf, $p\in M$.
\fl{Dann sind die folgenden Bedingungen äquivalent:}
\begin{enumerate}[1.]
\item $exp_p$ ist auf ganz $T_pM$ definiert
\item Abgeschlossene und beschränkte Teilmengen von $M$ sind kompakt.
\item $M$ ist als metrischer Raum vollständig.
\item $M$ ist geodätisch vollständig.
\end{enumerate}
\fl{Außerdem implizieren obige Bedingungen}
\begin{enumerate}[5.]
\item Zu jedem $q\in M$ gibt es eine Geodätische $\gamma$, die $p$ und $q$
verbindet, so dass $L(\gamma) = d(p,q)$.
\end{enumerate}

\begin{enumerate}[(i)]
\item \textsc{Bemerkung:} Aus $5. \not \Rightarrow 3.$:
Betrachte konvexe offene Teilmenge des $\R^n$
\item \textsc{Korollar:} Zwischen je zwei $p,q \in M$ (wie oben und vollständig)
existiert eine Geodätische der Länge $d(p,q)$. (Nicht eindeutig: Siehe $S^n$)
\item \textsc{Korollar:} $M$ (wie oben \& kompakt) ist vollständig.
\item \textsc{Bemerkung:} Auf nichtkompakten Riemn. Mf kann die Vollständigkeit von der gewählten Metrik abhängen.
\end{enumerate}
}
{}

%%%%%%%%%%%%%%%%%%%%%%%%%%%%%%%%%%%%%%%%%%%%%%%%%%%%%%%%%%%%%%%%%%%%%%%%%%%%%%%%
% 2.11 Isometrien und Killingfelder
%%%%%%%%%%%%%%%%%%%%%%%%%%%%%%%%%%%%%%%%%%%%%%%%%%%%%%%%%%%%%%%%%%%%%%%%%%%%%%%%
\kommentar{2.11 Isometrien und Killingfelder}

\nonameyet
{Definition 1} {Isometrie}
{
Sei $(M,g)$, $(N,h)$ pseudo-R. Mf und $\varphi : M \to N$ ein Diffeomorphismus.
\fl{Falls}
\[
h_{\varphi(p)}(d\varphi_p(v), d\varphi_p(w)) = g_p (v,w) 
\quad \forall p\in M, v,w \in T_pM,
\]
heißt $\varphi$ \texttt{Isometrie}. Diese Bedingung schreibt man auch
$\varphi^* h = g$.
\\
Existiert eine Isometrie zwischen $M$ und $N$, heißen $M$ und $N$ 
isometrisch.
}
{}

\nonameyet
{Lemma 1} {}
{
Sei $(M,g)$, $(N,h)$ pseudo-R. Mf und $\varphi : M \to N$ ein Isometrie. 
\fl{Dann gilt für alle Vektorfelder $X$ und $Y$ auf $M$,
dass }
\[
\varphi_*\nabla^M_X Y = \nabla^N_ {\varphi_* X} \varphi_* Y
\]
}
{}

\nonameyet
{Definiton 2} {homogen}
{
Eine pseudo-Riemannsche Mf heißt \texttt{homogen}, wenn es zu je zwei Punkten
$p,q \in M$ eine Isometrie $\varphi: S^n \to S^n$ gibt, so dass 
$\varphi(p) = q$.
}
{}

\nonameyet
{Definition 3} {Killingfelder}
{
Ein Vektorfeld $X$ auf einer pseudo-Riemannsche Mf heißt \texttt{Killingfeld},
falls die lokalen Flüsse von $X$ Isometrien sind.
\\
TODO Was heißt das konkret? 
}
{}

\nonameyet
{Definition 4} {Lie-Ableitung}
{
Sei $M$ eine diffb. MF, $X$ Vektorfeld auf $M$, $A$ ein Tensorfeld auf $M$ vom
Typ $(r,s)$.
\fl{Es sei $\Phi_t$ der lokale Fluss von $X$ auf einer offenen Menge 
$U \subset M$, dann definieren wir für $p \in U$ durch}
\[
(L_X A)_p = \frac{d}{dt}\bigg|_{t=1} (\Phi^* _t  A)_p
\]
die \texttt{Lie-Ableitung von $A$ entlang $X$.} 
Diese ist wieder ein Tensorfeld vom Typ $(r,s)$.
\\
TODO B10 Uebung 2
}
{}

\nonameyet
{Hauptsatz} {Äquivalenzen}
{
Sei $(M,g)$ pseudo-Riemannsche Mf mit Levi-Civita-Zusammenhang $\nabla$,
$X$ Vektorfeld auf $M$. 
\fl{Dann ist folgendes äquivalent:}
\begin{enumerate}[1.]
\item $X$ ist Killingfeld
\item $L_X g = 0$
\item $\langle \nabla_v X, w \rangle + \langle \nabla_w X, v\rangle = 0$
für alle $v,w \in T_pM$,$p\in M$
\end{enumerate}
}
{}
%%%%%%%%%%%%%%%%%%%%%%%%%%%%%%%%%%%%%%%%%%%%%%%%%%%%%%%%%%%%%%%%%%%%%%%%%%%%%%%%
% 2.12 Krümmung, Schnittkrümmung und Ricci-Krümmung
%%%%%%%%%%%%%%%%%%%%%%%%%%%%%%%%%%%%%%%%%%%%%%%%%%%%%%%%%%%%%%%%%%%%%%%%%%%%%%%%
\defreset
\satzreset

\kommentar{\small $\overset{\text{Riemmansche}}{\text{Krümmungstensor}}$}

\nonameyet
{Definition \thisdef und Satz} {\small Riemannscher Krümmungstensor}
{
\small
Sei $M$ differenzierbare Mf und $\nabla$ ein affiner Zusammenhang auf $M$.
Die Abbildung  
\[
R : \Gamma(TM) \times \Gamma(TM) \times \Gamma(TM) \to \Gamma(TM); 
\quad (X,Y,Z) \mapsto R(X,Y)Z
\]
\fl{definiert durch}
\[
R(X,Y)Z = \nabla_X \nabla_Y Z - \nabla_Y \nabla_X Z - \nabla_{[X,Y]} Z
\]
\fl{ist ein $(1,3)$-Tensorfeld. Er wird der 
\texttt{Riemannsche Krümmungstensor von $\nabla$} genannt.}
Ist $(M,g)$ eine pseudo-Riem. Mf mit Levi-Civita-Zusammenhang $\nabla$,
\fl{so nennt man $R$ auch \texttt{Riemannsche Krümmungstensor von $g$}.}
}
{}

\nonameyet
{Bemerkung} {\small Sichtweise auf $R$}
{
Mögliche Sichtweise auf den Riemannsche Krümmungstensor:
\fl{In lokalen Koordinaten eine beliebigen pseudo-Riem. Mf $(M,g)$ gilt}
\[
R(\pd[x_i],\pd[x_j])Z = \nabla_{\pd[x_i]}\nabla_{\pd[x_j]} Z - 
\nabla_{\pd[x_j]}\nabla_{\pd[x_i]} Z
\]
\fl{d.h, $R$ misst, inwieweit die obigen kovarianten Ableitungen miteinander 
kommutieren.}
}
{}

\nonameyet
{Lemma 1} {}
{
\small
$(M,g)$, $(N,h)$ seien pseudo-Riem. Mf mit entsprechenden 
Levi-Civita-Zusammenhängen $\nabla^M$ und $\nabla^N$ und dazu
assoziierten Riemannsche Krümmungstensoren $R^M$ und $R^N$.
\fl{Ist $\varphi: M \to N$ eine Isometrie, so gilt für alle 
$X,Y,Z \in \Gamma(TM)$:}
\[
\varphi_* (R^M(X,Y)Z) = R^N (\varphi_* X, \varphi_* Y) \varphi_* Z
\]
}
{}

\nonameyet
{Bemerkung 2} {}
{
Eine Riemannsche Mannigfaltikeit ist genau dann lokal isometrisch zu $\R^n$,
wenn der Krümmungstensor verschwindet.
}
{}

\nonameyet
{Satz \thissatz} {formale Eigenschaften von $R$}
{
\small
Sei $(M,\blf{\cdot,\cdot})$ eine pseudo-Riem. Mf mit assoziiertem 
Krümmungstensor $R$.
\fl{Dann gilt für alle $X,Y,Z,W \in \Gamma(TM)$:}
\begin{enumerate}[1.]
\item $R(X,Y)Z = -R(Y,X)Z$
\item $R(X,Y)Z + R(Y,Z)X + R(Z,X)Y  = 0$ \hfill (1. Bianchi-Identiät)
\item $\blf{R(X,Y)Z,W} = -\blf{R(X,Y)W,Z}$ 
\item $\blf{R(X,Y)Z,W} = \blf{R(Z,W)X,Y}$
\end{enumerate}
}
{}

\nonameyet
{Definition \thisdef} {algebraischer Krümmungstensor}
{
\small
$(V,\blf{\cdot,\cdot})$ sei ein endlich-dim. pseudo-Euklidscher VR.
\fl{Eine trilineare Abbildung}
\[
R: V\times V\times V\to V, \quad (u,v,w) \mapsto R(u,v)w,
\]
\fl{welche die vier formalen Eigenschaften des Riemannschen Krümmungstensors
erfüllt heißt \texttt{algebraischer Krümmungstensor}.}
\fl{Für den Krümmungstensor $R$ einer pseudo-Riem. Mf $(M,g)$ mit $p\in M$ sind 
also alle Abbildung:}
\[
R_p : T_pM \times T_pM \times T_pM \to T_pM 
\]
\fl{algebraischer Krümmungstensoren.}
}
{}
%%%%%%%%%%%%%%%%%%%%%%%%%%%%%%%%%%%%%%%%%%%%%%%%%%%%%%%%%%%%%%%%%%%%%%%%%%%%%%%%
% 2.12 Schnittkrümmung und Ricci-Krümmung
%%%%%%%%%%%%%%%%%%%%%%%%%%%%%%%%%%%%%%%%%%%%%%%%%%%%%%%%%%%%%%%%%%%%%%%%%%%%%%%%
\defreset
\satzreset
\kommentar{Schnittkrümmung \& Co}

\nonameyet
{Definition \thisdef und Lemma \thissatz} {2-Ebene,\small nichtentartet}
{
Sei $(V,\blf{\cdot,\cdot})$ ein pseudo-Eukl. VR. $\sigma \subset V$ sei
ein 2d-Unterraum, auch \texttt{2-Ebene} genannt,  und eine $\{u,v\}$ eine Basis.
\fl{Wir definieren}
\[
Q(u,v) = \blf{u,u}\blf{v,v} - \blf{u,v}^2.
\]
\fl{Falls $\blf{\cdot,\cdot}$ eingeschränkt auf $\sigma$ eine nichtentartete
symmetrische Bilinearform ist, so nennen wir auch \texttt{$\sigma$ 
nichtentartet}.} 
\fl{\textsc{Lemma \thesatz:} $\sigma$ ist genau dann nichtentartet, wenn 
$Q(u,v) \neq 0$.}
}
{}

\nonameyet
{\scriptsize Definition \thisdef und Lemma \thissatz} {\scriptsize Schnittkrümmung}
{
\small
Sei $(M,\blf{\cdot,\cdot})$ eine pseudo-Riem. Mf mit Levi-Civita-Zusammenhang
$\nabla$ mit assoziierten Krümmungstensor $R$. 
Sei $p\in M$ und $\sigma \subset T_pM$ eine nichtentartete 2-Ebene und 
$\{u,v\}$ eine Basis $\sigma$.
\fl{Die Zahl}
\[
K(\sigma) = \frac{\blf{R(u,v)v,u}}{Q(u,v)}
\]
\fl{definieren wir als die \texttt{Schnittkrümmung}.}
\fl{\textsc{Lemma \thesatz:} Diese Definition ist unabhängig von der Wahl der Basis
$\{u,v\}$.}
}
{}

\nonameyet
{\scriptsize Definition \thisdef und Beispiele} 
{\scriptsize konstante Schnittkrümmung}
{
Sei $(M,g)$ eine pseudo-Riemannsche Mf, $\kappa \in \R$

\fl{Gilt:}
\[
K(\sigma) = \kappa \quad \forall p\in M, \forall 
\text{ nichtentartete 2-Ebenen } \sigma \subset T_pM
\]
\fl{So sagen wir $M$ hat \texttt{konstante Schnittkrümmung}.}
\fl{\textsc{Beispiel:} Für $M=\R^n$ gilt $R =0$ und damit $K(\sigma) = 0$
unabhängig von $\sigma$.}
\fl{Gilt $\kappa = 0$ so heißt der Raum \texttt{flach}.}
}
{}

\nonameyet
{Lemma \thissatz und Beispiel} {$S^n$ Krümmung}
{
Es seien $M,N$ pseudo-Riemannsche Mf, $p\in M$, $\varphi: M \to N$ eine Isometrie,
$\sigma \subset T_pM$ eine 2-Ebene.
\fl{Dann gilt:}
\[
K^M(\sigma) = K^N(d\varphi_p(\sigma)).
\]
\fl{\textsc{Beispiel:} $S^n$, versehen mit Standardmetrik, hat konstante
Krümmung.}
}
{}

\nonameyet
{Lemma \thissatz} 
{ 
\begin{minipage}{0.43 \linewidth}
Approximation nichtentarteter Ebenen
\end{minipage}
}
{
Sei $(V,\blf{\cdot,\cdot})$ pseudo-Eukl. VR. Seien $u,v$ zwei linear
unabhängige Vektoren, die eine entarte 2-Ebene aufspannen.
\fl{Dann existiert zu Umgebung $U$ von $v$ ein $z\in U$, so dass 
$u$ und $z$ eine nichtentartete Ebene aufspannen.}
}
{}

\nonameyet
{Satz 1} {$R = R'$}
{
	\small
Es sei $(V,\blf{\cdot,\cdot})$ ein pseudo-Eukl. VR mit zwei Krümmungstensoren
$R$, $R'$ und dazugehörigen Schnittkrümmungen $K$, $K'$.
\fl{Gilt:}
\[
K(\sigma) = K'(\sigma) \quad 
\forall \text{ nichtentartete 2-Ebenen } \sigma \subset V,
\]
\fl{dann ist}
\[
R = R'
\]
}
{}

\nonameyet
{Satz 2} {Gestalt von $R$}
{
Es $(M,\blf{\cdot,\cdot})$ eine pseudo-Riem. Mf mit konstanter Schnittkrümmung 
$\kappa$.
\fl{Dann hat der Riemmansche Krümmungstensor von $g$ folgende Gestalt: }
\[
R(u,v)w = \kappa\cdot(\blf{v,w}u - \blf{u,w}v) .
\]
}
{}

\nonameyet
{Definition \thisdef} {pseudo ONB, Ricci-Tensor}
{
	\scriptsize
	\vspace{-3em}
Sei $(M,g)$ pseudo-Riem. Mf, $p\in M$ und $R$ ein Krümmungstensor.
\fl{Der \texttt{Ricci-Tensor} $Ric$ ist ein $(0,2)$ Tensorfeld,
definiert durch }
\[
Ric(X,Y)(p) = Spur(v\mapsto R_p(v,X(p))Y(p)).
\]
\fl{Ist $B:=\{e_1,\dots,e_n\}$ eine Basis von $T_pM$ für die gilt }
\[
\blf{e_i,e_i} = \pm \text{ für } i\in\{1,\dots,n\}
\quad \text{und} \quad 
\blf{e_i,e_j} = 0 \text{ für }  i \neq j
\]
\fl{so heißt $B$ \texttt{pseudo-ONB von $T_pM$}.}
\fl{Für so ein $B$ können wir $Ric$ schreiben als}
\[
Ric(X,Y)(p) 
= \sum_{i=1}^n 
\blf{e_i,e_i} \cdot \blf{R_p(e_i,X(p))Y(p),e_i}.
\]
\fl{Daraus liest man ab, das $Ric$ symmetrisch ist.}
\fl{Es gilt auch : $Ric = C^1_1(R)$ TODO}
\fl{\textsc{Intepretation 1:} Ist $v\in T_pM$, dann kann man $Ric(v,v)$ als
Mittel über die Schnittkrümmung aller 2-Ebenen, die $v$ enthalten, verstehen.}
}
{}

\nonameyet
{Definiton \thisdef} {Skalarkrümmung}
{
	\small
Sei $(M,\blf{\cdot,\cdot})$ pseudo-Riem. Mf, $p\in M$, $\{e_1,\dots,e_n\}$
pseudo ONB von $T_pM$.
\fl{Die \texttt{Skalarkrümmung} $scal\in C^\infty(M)$ ist definiert durch}
\[
scal(p) = \sum_{i=1}^n \blf{e_i,e_i} Ric(e_i,e_i).
\]
\fl{\textsc{Interpretation 1:} Mittel aller Schnittkrümmung in $p\in M$.}
}
{}
%%%%%%%%%%%%%%%%%%%%%%%%%%%%%%%%%%%%%%%%%%%%%%%%%%%%%%%%%%%%%%%%%%%%%%%%%%%%%%%%
% 3 Globale Riemannsche Geometrie
% 3.1 Jacobifelder
%%%%%%%%%%%%%%%%%%%%%%%%%%%%%%%%%%%%%%%%%%%%%%%%%%%%%%%%%%%%%%%%%%%%%%%%%%%%%%%%
\defreset
\satzreset
\kommentar{3.1 Jacobifelder}

\nonameyet
{Definition \thisdef und Lemma, Satz \thissatz} {\tiny Variationsvektorfeld}
{
	\scriptsize
	\vspace{-4em}
Sei $(M,g)$ eine Riemannsche Mf, $R$ der Krümmungstensor von $g$,
$c \in C^\infty([a,b],M)$, $\varepsilon > 0$.
\[
f: \underbrace{(-\varepsilon,\varepsilon) \times [a,b]}_{=: I} \to M,
\quad (s,t) \mapsto f(s,t) = f_s (t)
\quad \text{mit} \quad
f_0(t) = c(t) \; \forall t\in [a,b]
\]
\vspace{-2em}
\fl{heißt \texttt{Variation von c}. Es ist $f \in C^\infty(I,M)$.}
%~\\
Sei $V \in \Gamma_f(TM)$, also $V: (-\eps,\eps) \times [a,b] \to TM$ ist
glatt mit 
$V(s,t) \in T_{f(s,t)}M$.

%%%%
\fl{\textsc{Lemma:} Es gilt}
\vspace{-3em}
\[
\nablapt[s] \nablapt V 
= \nablapt \nablapt[s] V 
+ R(\frac{\partial f}{\partial s},\frac{\partial f}{\partial t}) V.
\]

Sei nun für alle $s\in (-\eps,\eps)$ die Kurve $f_s$ eine Geodätische. 
Insbesondere sei $f_0 =: \gamma$.
\fl{Wir definieren das Vektorfeld $J$ längs $\gamma$ 
namens \texttt{Variationsvektorfeld von $f$} mit}
\[
J : [a,b] \to TM, \; 
t \mapsto J(t) := \frac{\partial f}{\partial s} (s=0,t)
= \frac{\partial f}{\partial s}|_{s=0} (t) 
\]

%%%%
%\fl{\textsc{Satz:} } 
%\[
%\nabladt^2 J + R(J,\dot \gamma) \dot \gamma = 0 \qquad \text{(Jacobigleichung)}
%\]
\vspace{-2em}
\[
\text{\textsc{Satz 1:}} \quad 
J  \text{ erfüllt }
\underbrace{\nabladt^2 J + R(J,\dot \gamma) \dot \gamma = 0}
_{\text{\textsc{(Jacobigleichung)}}}
\; \Leftrightarrow \;
\exists \text{ Variation } f: I \to M \text { mit $J$ als Variationsvektorfeld.} 
\]

%%%%
\fl{\textsc{Definition:} Allgemein heißt ein Vektorfeld längs einer Geodätischen
$\gamma$, welches
die Jacobigleichung erfüllt, \texttt{Jacobifeld}.}
}
{}

\nonameyet
{Korollar zu Satz 1} {}
{
\small
Sei $(M,g)$ eine Riemannsche Mf, $R$ der Krümmungstensor von $g$, 
$\gamma : [0,b] \to M$ sei eine Geodätische und 
$J$ ein Jacobifeld längs $\gamma$ mit $J(0) = 0$, $v = \dot \gamma(0)$,
$w = \nabladt J (0)$.
\fl{Dann gilt}
\[
J(t) = (d \exp_p)_{tv}(tw)
\]
}
{}
\nonameyet
{Satz \thissatz} {}
{
	\small 
Sei $(M,g)$ n-dim. Riem. Mf, $\gamma : [a,b] \to M$ eine Geodätische in $M$.
\fl{Dann bilden die Jacobifelder längs $\gamma$ einen $2n$-dim Vektorraum.
Genauer:}
\[
\forall v,w \in T_{\gamma(a)}M \quad \exists ! \text{ Jacobifeld } J: 
\quad J(a) =  v ,\; \nabladt J (b) = w
\]
\[
I : \{\text{Jacobifelder längs $\gamma$}\} \to T_pM \oplus T_pM,
\quad J \mapsto (J(a),\nabladt J(b)) 
\quad \text{ ist ein Isomorphismus}.
\]
(TODO: Lösung von DGL bilden Vektorraum)
}
{}

\nonameyet
{Beispiele 1} {}
{
	\small
\begin{enumerate}[1.]
\item $\dot \gamma$ immer Jacobifeld
\item $J(t) = t \dot\gamma(t)$ Jacobifeld
\item $J(t) = f(t) \dot \gamma(t)$
\item Sei $\gamma\in C^\infty([0,a],M)$. Betrachte die Variation 
$f(s,t) = \text{exp}_p(tv(s))$, wobei $v(s)$ Kurve in $T_pM$ ist.
Setze $v = v(0)$ und $w = v'(0)\in T_pM$. 
Für alle $s$ ist $f_s \in C^\infty$ eine Geodätische, also folgt
\[
J(t) = \frac{\partial f}{\partial s} (0,t) = (d \exp_p)_{tv}(tw)
\]
ein Jacobifeld längs $\gamma_v$ mit $J(0) = 0$ ist.
\end{enumerate}
}
{}

\nonameyet
{Satz \thissatz } {}
{
	\small
Sei $M$ Riemannsche Mf mit konstanter Krümmung $\kappa$.
Es sei $\gamma : [0,a] \to M$ eine nach Bogenlänge parametrisierte Geodätische
in $M$. 
\\ 
Weiter sei $J$ ein Jacobifeld längs $\gamma$ mit $J(0) = 0$ 
und $J(t) \perp \dot c(t)$.
\\
Weiterhin sei $X$ das eindeutige parallele Vektorfeld längs $\gamma$ mit 
$X(0) = \nabladt J (0)$.
\fl{Dann gilt:}
\[
J(t) =
\begin{cases}
\frac{\sin(t\sqrt{\kappa})}{\sqrt{k}} X(t)  & \kappa > 0\\
tX(t) & \kappa = 0\\
\frac{\sinh(t\sqrt{-\kappa})}{\sqrt{-k}} X(t)  & \kappa < 0\\
\end{cases}
\]
}
{}

%%%%%%%%%%%%%%%%%%%%%%%%%%%%%%%%%%%%%%%%%%%%%%%%%%%%%%%%%%%%%%%%%%%%%%%%%%%%%%%%
% 3.2 Konjungierte Punkte
%%%%%%%%%%%%%%%%%%%%%%%%%%%%%%%%%%%%%%%%%%%%%%%%%%%%%%%%%%%%%%%%%%%%%%%%%%%%%%%%
\defreset
\satzreset
\kommentar{3.2 Konjugierte Punkte}


\nonameyet
{Definiton \thisdef} {\scriptsize Konjugierte Punkte,Vielfachheit}
{
\small
Sei $(M,g)$ eine $n$-dim. Riemannsche Mf, $\gamma : [a,b] \to M$ eine nichtkonstante
Geodätische in $M$
mit $\gamma(a) = p$ und $\gamma(b) = q$.
\fl{Existiert ein Jacobifeld $J$ entlang $\gamma$ mit $J(a) = 0$ und 
$J(b) = 0$, das nirgends verschwindet, so sagen wir }
\texttt{$q$ ist entlang $\gamma$ zu $p$ konjugiert} \textit{oder auch} 
$p,q$ sind \texttt{konjugierter Punkt entlang $\gamma$}.
\fl{Sei $\gamma$ wie oben, dann gilt:}
\[
\left\{
\text{$J$ ist Jacobifeld entlang $\gamma$} 
\;|\; 
\text{$q$ ist entlang $\gamma$ zu $p$ konjugiert}
\right \}
\]
\fl{ist ein $(n-1)$-dim. Vektorraum. Diese Dimension wird auch 
\texttt{Vielfachheit des konjugierten Punktes $q$} genannt.}
\fl{\textsc{Beipiel 1:} $S^n$ TODO}
\fl{\textsc{Beipiel 2:} $\kappa <= 0$ TODO}
}
{}

\nonameyet
{Satz \thissatz} {}
{
\small
$(M,g)$ $n$-dim. Riemannsche Mf, $\gamma: [0,a] \to M$ eine Geodätische mit 
$p = \gamma(0)$, $v = \dot \gamma(0)$.
\fl{Dann gilt}
\begin{enumerate}[1.]
\item  Der Punkt $\gamma(t_0)$ für $t_0\in [0,a]$ ist genau dann entlang $\gamma$
zu $p$ konjungiert, wenn $t_0v$ ein kritischer Punkt von $\exp_p$ ist.
\item Die Vielfachheit von $\gamma(t_0)$ ist gleich der Dimension des 
Kerns von $(d\exp)_p(t_0v)$.
\end{enumerate}

}
{}

%%%%%%%%%%%%%%%%%%%%%%%%%%%%%%%%%%%%%%%%%%%%%%%%%%%%%%%%%%%%%%%%%%%%%%%%%%%%%%%%
% 3.3 Der Satz von Hadamard
%%%%%%%%%%%%%%%%%%%%%%%%%%%%%%%%%%%%%%%%%%%%%%%%%%%%%%%%%%%%%%%%%%%%%%%%%%%%%%%%
\defreset
\satzreset
\kommentar{3.3 Satz von Hadamard}

\nonameyet
{Satz \thissatz} {Version 1}
{
$(M,g)$ sei vollständige Riemannsche Mannigfaltikeit mit nichtpositiver
Schnittkrümmung, d.h. $K(\sigma) \leq 0$ für alle 2-Ebenen 
$\sigma \subset T_pM$, $\forall p\in M$.
\fl{Dann gibt es keine konjugierten Punkte in $M$, d.h.}
\[
\exp_p : T_pM \to M \quad \forall p\in M
\]
ist ein lokaler Diffeomorphismus auf ganz $T_pM$.
}
{}

\nonameyet
{Satz \thissatz} {Version 2}
{
Sei $(M,g)$ zusammenhängende und vollständige Riem. Mf mit nichtpositiver
Schnittkrümmung.
\fl{Dann ist }
\[
\exp_p : T_pM \to M \quad \forall p\in M
\]
\fl{eine differenzierbare Überlagerung.}
}
{}

\nonameyet
{Satz \thissatz} {Version 3, Cartan-Hadamard}
{
Sei $(M,g)$ vollständige, einfach zusammenhängende Riem Mf mit
nichtpositiver Schnittkrümmung.
\fl{Dann ist }
\[
\exp_p : T_pM \to M \quad \forall p\in M
\]
\fl{ein Diffeomorphismus.}
\fl{Insbesondere existiert für beliebige Punkt $p,q\in M$, genau eine 
Geodätische von $p$ nach $q$.}
}
{}

%%%%%%%%%%%%%%%%%%%%%%%%%%%%%%%%%%%%%%%%%%%%%%%%%%%%%%%%%%%%%%%%%%%%%%%%%%%%%%%%
% 3.5 Variation der Energie
%%%%%%%%%%%%%%%%%%%%%%%%%%%%%%%%%%%%%%%%%%%%%%%%%%%%%%%%%%%%%%%%%%%%%%%%%%%%%%%%
\defreset
\satzreset
\kommentar{3.5 Variation der Energie}

\nonameyet
{\small Definition \thisdef \& Notiz} {\small Energiefunktional}
{
\footnotesize
\vspace{-2em}
Sei $(M,g)$ ein Riemannsche Mf, $c \in C^\infty([0,a],M)$ mit 
Länge 
\[
L(c) = \int_0^a \| \dot c(t) \| dt.
\]
Sei weiter $\eps > 0$ und $f : (-\eps,\eps) \times [a,b] \to M$ eine 
Variation von $c$.
\fl{Wir nennen die Abbildung $E$}
\[
E : \Omega_{0,a} \to \R, \; c \mapsto E(c) = \int_0^a \|\dot c(t)\|^2 dt 
\]
\fl{das \texttt{Energiefunktional}.}

%%%%
\fl{\textsc{Notiz 1:} Weiter gilt mit der Cauchy-Schwarz-Ungleichung:}
\[
\left(\int_0^a \|\dot c(t) \| \cdot 1 dt \right) ^2
= \quad  
\fbox{$L(c)^2 \leq aE(c) $}
\quad = 
\left(\int_0^a \|\dot c(t) \|^2 dt \right) \cdot \left(\int_0^a 1 dt \right) 
\]
\fl{Mit Gleichheit genau dann, wenn $c$ proportional zur Bogenlänge 
parametrisiert ist}

%%%%
%\scriptsize
\fl{\textsc{Notiz 2:} Aussagen in 3.5 gelten z.T. auch für $L$, falls 
$c$ eine reguläre Kurve ist.} 
}
{}

\nonameyet
{Lemma 1} {}
{
\small
Sei $(M,g)$ Riemannsche Mf, $p,q \in M$ sowie $\gamma :[a,b] \to M$ eine die 
Länge minimierende Geodätische von $p$ nach $q$.
\fl{Dann gilt für alle $c\in C^\infty([0,a],M)$ von $p$ nach $q$:}
\[
E(\gamma) \leq E(c).
\]
\fl{Mit Gleichheit genau dann, wenn $c$ eine die Länge minimierende Geodätische 
ist.}
}
{}

\nonameyet
{Lemma 2 und Definition} {eigentlich}
{
Sei $(M,g)$ Riemannsche Mf, $c \in C^\infty([0,a],M)$, $V \in \Gamma_c(TM)$
und $\eps > 0$.
\fl{Dann existiert eine Variation $f$}
\[
f : (-\eps,\eps) \times [0,a] \to M
\]
\fl{mit Variationsvektorfeld $V$.}
Ist noch $V(0) = 0$ und $V(a) = 0$, 
\\
\fl{so können wir annehmen, dass }
\[
f(s,0) = c(0) \text{ und } f(s,a) = c(a) \quad \forall s \in (-\eps,\eps).
\]
\fl{Wir sagen dazu, dass die Variation $f$ \texttt{eigentlich} ist.}
}
{}

\nonameyet
{\scriptsize Satz \thissatz \& Korollar} 
{ \begin{minipage}{0.28\linewidth}
\scriptsize Erste Variationsformel\\ für die Energie
\end{minipage}}
{
	\small
$(M,\blf{\cdot,\cdot})$ eine Riemannsche Mf, $c\in C^\infty([0,a],M)$, 
$\eps > 0$,$f$ eine 
Variation von $c$ mit $f_s(t)\ C^\infty([0,a],M)$ für alle $s\in (-\eps,\eps)$. 
Es sei weiter  
\[
E(s) := E(f_s) = \int_0^a \| \dot f_s(t)\|^2 dt
\]
und $V$ bezeichne das Variationsvektorfeld der Variation $f$.
\fl{Dann gilt:}
\[
\frac{1}{2} \dot E(0) = - \int_0^a \blf{V(t),\nabladt \dot c (t)}dt 
- \blf{V(0),\dot c(0)} + \blf{V(a), \dot c(a)}.
\]

\fl{\textsc{Korollar:} $c\in C^\infty([0,a],M)$ ist genau dann eine Geodätische,
wenn für jede eigentliche Variation $f$ von $c$ gilt, dass $\dot E(0) = 0$.}
}
{}
\newcommand{\ddt}[1]{\overset{\;\textbf{..}}{#1}}
\newcommand{\nabladdt}{\frac{\nabla^2}{dt^2}}
\nonameyet
{\scriptsize Satz \thissatz \& Notiz} 
{ \begin{minipage}{0.28\linewidth}
\scriptsize Zweite Variationsformel\\ für die Energie
\end{minipage}}
{
	\small
$(M,\blf{\cdot,\cdot})$ eine Riemannsche Mf mit $R$ als Krümmungstensor von 
$\blf{\cdot,\cdot}$, $c\in C^\infty([0,a],M)$, $\eps > 0$, $f$ eine 
Variation von $c$ mit $f_s(t)\ C^\infty([0,a],M)$ für alle $s\in (-\eps,\eps)$. 
$V$ bezeichne das Variationsvektorfeld der Variation $f$.
\fl{Dann gilt:}
\[
\begin{aligned}
\frac{1}{2} \ddt{E}(0) =& 
- \int_0^a \blf{\nabladdt V + R(V,\dot c)\dot c, V}(t) dt 
\\ &
- \blf{\nabladt[s] \frac{\partial f}{\partial s}, \dot c}(0,0)
+ \blf{\nabladt[s] \frac{\partial f}{\partial s}, \dot c}(0,a)
\\ &
- \blf{V(0),\nabladt V(0)}|_{t=0} + \blf{V(a), \nabladt V(a)}|_{t =a}.
\end{aligned}
\]

\fl{\textsc{Notiz 1:} Ist $f$ eine eigentliche Variation von $c$, fallen alle 
bis auf den ersten Summanden weg.}

\fl{\textsc{Notiz 2:} Der erste Eintrag in $\blf{\cdot,\cdot}$ im ersten 
Summanden ist gerade die Jacobigleichung.}
}
{}

%%%%%%%%%%%%%%%%%%%%%%%%%%%%%%%%%%%%%%%%%%%%%%%%%%%%%%%%%%%%%%%%%%%%%%%%%%%%%%%%
% 3.6 Bonnet-Myers
%%%%%%%%%%%%%%%%%%%%%%%%%%%%%%%%%%%%%%%%%%%%%%%%%%%%%%%%%%%%%%%%%%%%%%%%%%%%%%%%
\defreset
\satzreset
\kommentar{Bonnet-Myers}

\nonameyet
{Satz} {Bonnet-Myers}
{
\small
Sei $(M,\blf{\cdot,\cdot})$ eine vollständige Riemannsche Mannigfaltigkeit. 
Es gebe ein $r > 0$, so dass
\[
Ric_p(v,v) \geq \frac{n-1}{r^2}\blf{v,v} > 0 
\quad \forall p\in M,
\; \forall v\in T_pM 
%\text{ mit } \|v\| = 1.
\]
\fl{Dann ist $M$ kompakt und der \texttt{Durchmesser} von $M$}
\[
\text{diam}(M) := \sup_{p,q \in M} d(p,q) 
\]
\fl{erfüllt }
\[
\text{diam}(M) \leq \pi r.
\]
\fl{Weiterhin ist die Fundamentalgruppe $\pi_1(M)$ ist endlich.}
}
{}

%%%%%%%%%%%%%%%%%%%%%%%%%%%%%%%%%%%%%%%%%%%%%%%%%%%%%%%%%%%%%%%%%%%%%%%%%%%%%%%%
% 5. Riemannsche Untermannigfaltigkeiten
% 5.1 Zweite Fundamentalform
%%%%%%%%%%%%%%%%%%%%%%%%%%%%%%%%%%%%%%%%%%%%%%%%%%%%%%%%%%%%%%%%%%%%%%%%%%%%%%%%
\defreset
\satzreset
\kommentar{5.1 Zweite Fundamentalform}
\newcommand{\M}[1][M]{\overline{#1}}


\nonameyet
{Setup } {1/2}
{
\scriptsize
Sei $M$ $n$-dim. Mf, $(\M,\M[g]=\blf{\cdot,\cdot})$ Riemannsche Mf der Dimension $n+k$.
Sei $i : M \to \M$ eine Immersion. Mittels 
\vspace{-1em}
\[
\blf{v,w} := \blf{di_p(v),di_p(w)} \quad \text{ für } p\in M, \; v,w \in T_pM
\vspace{-1em}
\]
\fl{zieht man die Metrik von $\M$ auf $M$ zurück 
(\texttt{isometrischen Immersion}).}
\fl{Alle Betrachtungen sind lokal und damit kann o.B.d.A. $i$ als Einbettung 
angenommen werden. In diesem Sinn betrachten wir $M \subset \M$, da 
$i(M) \overset{\sim}{=} M$.}

\fl{Sei $(U,\varphi)$ Unter-Mfskarte mit $\varphi = (x_1,\dots,x_{n+k})$
und $(U\cap M,\psi)$ mit $\psi:=(x_1,\dots,x_n)$ eine Karte von $M$. 
$\varphi$ sei sogewählt, das $\varphi(U)$ invariant unter der Projektion
$\pi : \R^{n+k} \to \R^n$ ist. Mithilfe dieser Karten können wir eine 
gegebene Funktion $f: M \to \R$ auf $U$ Fortsetzen der Einschränkung 
$f|_{U\cap M}:$}
\[
\M[f] = f\circ \psi ^{-1} \circ \pi \circ \varphi
\]
}
{}

\nonameyet
{Setup } {2/2}
{
	\scriptsize
\fl{Ähnlich kann man mit lokalen Vektorfelder auf $M$ verfahren. 
Es seien $(\pd[x_1],\dots,\pd[x_{n+k}])$ die zu $\varphi$ assoziierten lokalen
Basisfelder. Schränken wir diese auf $U\cap M$ ein, so hat man die lokalen 
Basisfelder der Karte $\psi$ von $M$. Für $X\in \Gamma(TU\cap M)$ können wir
schreiben}
\[
X = \sum_{i=1}^n f_i \pd[x_i]\bigg|_{U\cap M}
\]
Und erhalten für 
\[
\M[X] := \sum_{i=1}^n \M[f]_i \pd[x_i] \quad \text{ein } \M[X] \in \Gamma(TU)
\text{ mit }
\M[X]|_M = X .
\]
\fl{Dies kann auch so ausdrücken}
\[
\M[X] \circ i = di(X),
\]
d.h. $X$ und $\M[X]$ sind $i$-verwandt.
}
{}

\nonameyet
{\scriptsize Definition} {\scriptsize Normalenraum,2.Fundamentalform}
{
\scriptsize
\vspace{-2em}
Alles wie im setup. Sei $p\in M \subset \M$. 
Mittels Isomorphie sei $T_pM = di(T_pM)$ 
\fl{Wir definieren den \texttt{Normalenraum} von $M$ in $\M$ in $p$ durch}
\[
\nu_p M := \{ v\in T_p\M | v \perp T_pM \}.
\]
\fl{Wir nennen $\nu M := \sqcup_{p\in M} \nu_p M$ das \texttt{Normalenbündel}.}
%\[
%\nu M : = \bigcup_{p\in M} \nu_p M 
%\text{ trägt die Struktur eines Vektorbündels über $M$.}
%\]
\fl{Betrachte die Zerlegung:}
\vspace{-2em}
\[
T_p\M = T_p M \otimes \nu_p M 
\text{ und } T_pM \ni v = 
\overbrace{v^{\top}}^{\text{tang. Anteil}} 
+ \overbrace{v^{\perp}}^{\text{normalen Anteil}}
\]
\fl{Seien $X,Y \in \Gamma(TU\cap M)$ beliebig fortgesetz zu
$\M[X],\M[Y] \in \Gamma(TU)$ und betrachte }
\[
\M[\nabla]_{\M[X]} \M[Y] = (\M[\nabla]_{\M[X]} \M[Y])^\top +
(\M[\nabla]_{\M[X]} \M[Y])^\perp.
\]
\fl{Wir definieren die \texttt{zweite Fundamentalform $\alpha$ von M} durch}
\[
\alpha(X,Y)(p) := (\M[\nabla]_{\M[X]_p} \M[Y])^\perp
\]
\fl{\textsc{Lemma:} Die zweite Fundamentalform $\alpha$ ist wohldefiniert,
symmetrisch in $X$ und $Y$ und $C^\infty$-linear.}
\fl{\textsc{Gauß-Formel:}}
\vspace{-2em}
\[
(\M[\nabla]_{\M[X]} \M[Y])(p) = (\M[\nabla]_{X} Y)(p) 
= (\nabla_X Y)(p) + (\alpha(X,Y))(p)
\]
}
{}

\nonameyet
{\small Definitionen und Sätze} {\scriptsize totalgeodätisch}
{
\scriptsize
\fl{Es gilt}
\vspace{-2.5em}
\[
\frac{\M[\nabla]}{dt} Y = \nabladt Y + \alpha(\dot c, Y)
\]

\fl{\textsc{Definition:} Eine Unter-Mf $M \subset \M$ heißt
\texttt{totalgeodätisch in $\M$}, wenn 
$\forall p\in M, \;\forall v\in T_pM$
die Geodätische in $\M$ durch $p$ in Richtung $v$ komplett in $M$ 
verläuft. 
\par Dies gilt genau dann, wenn die zweite Fundamentalform verschwindet.}

\fl{\textsc{Weingarten-Abbildung:} Für die Erweiterung 
$\M[\xi] \in \Gamma(\nu U)$ gibt lokal folgendes Sinn}
\vspace{-1em}
\[
\blf{\M[\nabla]_X \xi, Y} = - \blf{\xi,\alpha(X,Y)}.
\]

\fl{\textsc{Gauß-Gleichung:} Für alle $x,y,z,w \in T_pM$ gilt}
\[
\blf{\M[R](x,y)z,w} = \blf{R(x,y)z,w} - \blf{\alpha(x,w),\alpha(y,z)}
+ \blf{\alpha(x,z),\alpha(y,w)}
\]

\fl{\textsc{Korollar}: Hat $\M$ kostanten Schnittkrümmung $\kappa$ so gilt}
\vspace{-1em}
\[
K(\sigma) = \kappa + \blf{\alpha(x,x),\alpha(y,y)} - \|\alpha(x,y)\|^2
\]
\fl{Ist $M$ also totalgeodätisch in $\M$, dann hat $M$ ebenfalls konst. 
Schnittk.}
}
{}

\nonameyet
{Definition} {Weingartenoperator}
{
	\small
\fl{Wir definieren den Weingartenoperator $A_\xi$ durch}
\[
A_\xi X = -(\M[\nabla]_X \xi)^\top.
\]
\fl{Damit wird aus der der Weingartengleichung}
\[
\blf{A_\xi X,Y} = \blf{\xi,\alpha(X,Y)}.
\]
\fl{Zu dem ist $A_\xi :T_pM \to T_pM$ ein linearer selbstadjungierter 
Operator. Dieser ist orthogonal diagonalisierbar. Die Eigenvektoren
nennen wir \texttt{Hauptkrümmungsrichtung} und die Eigenwerte 
\texttt{Hauptkrümmung}.}
\tiny 
\fl{\textsc{Was fehlt:}  Beispiel,Normalenzusammenhang,
		komische Abbildung,
		Codazzi-Gleichung,
		Bemerkung, 5.2.}
}
{}
%%%%%%%%%%%%%%%%%%%%%%%%%%%%%%%%%%%%%%%%%%%%%%%%%%%%%%%%%%%%%%%%%%%%%%%%%%%%%%%%
% ENDE  ENDE   ENDE  ENDE  ENDE  ENDE  ENDE  ENDE  ENDE  ENDE  ENDE  ENDE
%%%%%%%%%%%%%%%%%%%%%%%%%%%%%%%%%%%%%%%%%%%%%%%%%%%%%%%%%%%%%%%%%%%%%%%%%%%%%%%%

\end{document}
%% Template	
\kommentar{\scriptsize K}
\begin{karte}[]{%
			Inhalt	
}
\centr
\vspace{-3.1em}
%Rückseite
\centr
\end{karte}


